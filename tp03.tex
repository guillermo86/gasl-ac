%%% LaTeX Template: Article/Thesis/etc. with colored headings and special fonts
%%%
%%% Source: http://www.howtotex.com/

\documentclass[12pt]{article}


\usepackage{apuntes-estilo}
\usepackage{fancyhdr,lastpage}
\usepackage{verbatim}



\def\maketitle{

% Titulo 
 \makeatletter
 {\color{bl} \centering \huge \sc \textbf{
Trabajo práctico N 3 \\
\large \vspace*{-8pt} \color{black} Arbol de Directorios
 \vspace*{8pt} }\par}
 \makeatother


% Autor
 \makeatletter
 {\centering \small 
	Introducción a la Administración de Sistemas \\
 	Departamento de Ingeniería de Computadoras \\
 	Facultad de Informática - Universidad Nacional del Comahue \\
 	\vspace{20pt} }
 \makeatother

}

% Custom headers and footers
\fancyhf{} % clear all header and footer fields
\fancypagestyle{plain}{\fancyhf{}}
  	\pagestyle{fancy}
 	\lhead{\footnotesize TP N 3 - Arbol de Directorios }
 	\rhead{\footnotesize \thepage\ }	% "Page 1 of 2"

\def\ti#1#2{\texttt{#1} & #2 \\ }



\begin{document}

\thispagestyle{empty}
\maketitle
\setlength{\parindent}{0pt}

\paragraph{Los siguientes ejercicios se realizan en su totalidad en el equipo asignado al grupo.
Utilice la redirección y el editor \texttt{vi} para guardar las salidas y editar los resultados 
en un archivo llamado trabajo-practico-3.txt . En el archivo de resolución indique su nombre y apellido, y el número de ejercicio a resolver, junto a la salida incluyendo el comando que ejecuta. }

\section{Ejercicio 1.}

\begin{enumerate}
\item Sobre el directorio /etc
\begin{itemize}
\item - ¿Qué tipo de archivos hay en ese directorio?
\end{itemize}

\item El archivo /etc/passwd contiene el listado de usuarios del sistema.
Cada linea de texto en ese archivo define los detalles de un unico usuario.

\begin{itemize}
\item - Utilice cat para visualizar el contenido de /etc/passwd, en la pantalla.
\item - Utilice cat y redirija la salida al archivo en donde coloca los resultados del practico.
\item - ¿Cuántos usuarios hay definidos en el sistema?
Ayuda: Utilice el comando cat redirigiendo la salida a wc (word count) para contar cuantas lineas
existen en el archivo.
\end{itemize}


\item El archivo /etc/shadow contiene las passwords de los usuarios de manera encriptada.
\begin{itemize}
\item - ¿Que permisos tiene ese archivo? Utilice el comando ls y redirija la salida a trabajo-practico-3.txt
\item - ¿Quien puede leer y modificar ese archivo?
\item - ¿Puede un usuario comun modificar /etc/shadow? ¿Explique porqué?
\end{itemize}


\item El archivo /etc/debian\_version contiene en texto plano el nombre de la
versión de Debian Linux instalada.
\begin{itemize}
\item - ¿Qué versión de Debian utiliza en la práctica?
\end{itemize}
\end{enumerate}


\section{Ejercicio 2.}
Sobre el directorio /usr/bin

\begin{itemize}
\item - ¿Qué tipo de archivos hay en ese directorio?
\item - ¿Cuáles son los permisos del directorio /usr/bin?
\item - ¿Quien puede modificar el contenido de ese directorio?
\item - ¿Por qué el directorio debe tener esos permisos, y no otros?
\end{itemize}


\section{Ejercicio 3.}
El comando du se puede utilizar para ver el espacio utilizado por archivos y directorios.
Si ejecuta du -h /etc el sistema muestra de manera legible (el argumento -h significa "legible para las personas")
el espacio utilizado en el directorio /etc (suma los tamaños de todos los archivos y directorios que estén dentro de /etc).

\begin{itemize}
\item - ¿Cuanto espacio utiliza el directorio /usr?
\item - ¿Porqué el directorio /usr ocupa mas espacio que /etc?
\item - ¿Cuanto espacio utiliza el directorio raiz / ?
\end{itemize}


\section{Ejercicio 4.}
\begin{enumerate}
\item El comando hostname se utiliza para conocer el nombre del sistema (o para definirlo).
\begin{itemize}
\item - ¿Qué nombre tiene su sistema?
\end{itemize}
\end{enumerate}

\begin{enumerate}
\item El programa hostname viene empaquetado en un paquete de nombre homónimo.
El comando dpkg -L hostname  lista los archivos que el paquete hostname instaló en el sistema.
\begin{itemize}
\item - Indique que son cada uno de los archivos listados (por ejemplo: si son binarios ejecutables, archivos de configuracion, archivos de log, etc).
\end{itemize}
\end{enumerate}



\section{Ejercicio 5.}
El archivo /proc/meminfo contiene información de la memoria utilizada en kilobytes.

\begin{itemize}
\item - Utilice cat para ver la información.
\item - ¿Cuanta memoria total en MegaBytes tiene el sistema?
\item - ¿Cuanta memoria se está utilizando del total? (en MegaBytes). Ayuda: utilice el dato MemFree
\end{itemize}


\section{Ejercicio 6.}
En el directorio /proc existen directorios con nombres "numéricos", que corresponden con el PID de los procesos del sistema.
Estos directorios contienen información de los procesos (de los programas en ejecución).
\begin{itemize}
\item - Liste los directorios en /proc, en el cual, su nombre, comience con un número (liste solo los nombres de los directorios, no su contenido). Ayuda : opción -d de ls.
\item - Cuente con wc -l el listado anterior. ¿Cuántos procesos tiene en ejecucion el sistema?.
\item - Ejecute ps -ef , y ps -ef | wc -l . ¿Qué reporta ps -ef?
\item - ¿Coincide la cantidad de procesos reportados con ps -ef con el obtenido en el primer inciso de este ejercicio?
\end{itemize}


\section{Ejercicio 7.}
El comando find /var lista todos los archivos y directorios que exiten debajo de /var.
\begin{itemize}
\item - ¿Cuantos archivos y directorios contiene el /var de su sistema?. Ayuda : wc
\item - ¿Cuanto espacio utiliza en disco todo el /var?. Ayuda: du
\item - ¿Cuál es el directorio en donde el sistema guarda los mensajes de log?
\item - Mencione 5 archivos de log de su sistema.
\end{itemize}








\end{document}
