%%% LaTeX Template: Article/Thesis/etc. with colored headings and special fonts
%%%
%%% Source: http://www.howtotex.com/

\documentclass[12pt]{article}


\usepackage{apuntes-estilo}
\usepackage{fancyhdr,lastpage}
\usepackage{verbatim}



\def\maketitle{

% Titulo 
 \makeatletter
 {\color{bl} \centering \huge \sc \textbf{
Trabajo práctico N 5 \\
\large \vspace*{-8pt} \color{black} Inicio y apagado, repaso. 
 \vspace*{8pt} }\par}
 \makeatother


% Autor
 \makeatletter
 {\centering \small 
	Introducción a la Administración de Sistemas \\
 	Departamento de Ingeniería de Computadoras \\
 	Facultad de Informática - Universidad Nacional del Comahue \\
 	\vspace{20pt} }
 \makeatother

}

% Custom headers and footers
\fancyhf{} % clear all header and footer fields
\fancypagestyle{plain}{\fancyhf{}}
  	\pagestyle{fancy}
 	\lhead{\footnotesize TP N 5 - Inicio y apagado, repaso. }
 	\rhead{\footnotesize \thepage\ }	% "Page 1 of 2"

\def\ti#1#2{\texttt{#1} & #2 \\ }



\begin{document}

\thispagestyle{empty}
\maketitle
\setlength{\parindent}{0pt}

\paragraph{Los siguientes ejercicios se realizan en su totalidad en el equipo asignado al grupo.
Utilice la redirección y el editor \texttt{vi} para guardar las salidas y editar los resultados 
en un archivo llamado trabajo-practico-5.txt . En el archivo de resolución indique su nombre y apellido, 
y el número de ejercicio a resolver, junto a la salida incluyendo el comando que ejecuta. }

\section{Ejercicio 1}

\begin{enumerate}
\item El gestor de arranque utilizado por las máquinas del laboratorio se llama GRUB. Investigue 
dónde se encuentran los archivos de configuración del mismo dentro del sistema. 
\item Explique con sus palabras cuál es la función de dicho software.
\end{enumerate}


\section{Ejercicio 2}
Sobre el directorio \texttt{/usr/bin}

\begin{itemize}
\item - ¿Qué tipo de archivos hay en ese directorio?
\item - ¿Cuáles son los permisos del directorio \texttt{/usr/bin}?
\item - ¿Quien puede modificar el contenido de ese directorio?
\item - ¿Por qué el directorio debe tener esos permisos, y no otros?
\end{itemize}


\section{Ejercicio 3.}
El comando \texttt{du} se puede utilizar para ver el espacio utilizado por archivos y directorios.
Si ejecuta \texttt{du -h /etc} el sistema muestra de manera legible (el argumento \texttt{-h} significa "legible para las personas")
el espacio utilizado en el directorio \texttt{/etc} (suma los tamaños de todos los archivos y directorios que estén dentro de \texttt{/etc}).

\begin{itemize}
\item - ¿Cuanto espacio utiliza el directorio \texttt{/usr}?
\item - ¿Porqué el directorio \texttt{/usr} ocupa mas espacio que \texttt{/etc}?
\item - ¿Cuanto espacio utiliza el directorio raíz \texttt{/} ?
\end{itemize}


\section{Ejercicio 4.}
\begin{enumerate}
\item El comando \texttt{hostname} se utiliza para conocer el nombre del sistema (o para definirlo).
\begin{itemize}
\item - ¿Qué nombre tiene su sistema?
\end{itemize}
\end{enumerate}

\begin{enumerate}
\item El programa \texttt{hostname} viene empaquetado en un paquete de nombre homónimo.
El comando \texttt{dpkg -L hostname}  lista los archivos que el paquete \texttt{hostname} instaló en el sistema.
\begin{itemize}
\item - Indique que son cada uno de los archivos listados (por ejemplo: si son binarios ejecutables, archivos de configuración, archivos de log, etc).
\end{itemize}
\end{enumerate}



\section{Ejercicio 5.}
El archivo \texttt{/proc/meminfo} contiene información de la memoria utilizada en kilobytes.

\begin{itemize}
\item - Utilice cat para ver la información.
\item - ¿Cuanta memoria total en MegaBytes tiene el sistema?
\item - ¿Cuanta memoria se está utilizando del total? (en MegaBytes). Ayuda: utilice el dato MemFree
\end{itemize}


\section{Ejercicio 6.}
En el directorio \texttt{/proc} existen directorios con nombres "numéricos", que corresponden con el PID de los procesos del sistema.
Estos directorios contienen información de los procesos (de los programas en ejecución).
\begin{itemize}
\item - Liste los directorios en \texttt{/proc}, en el cual, su nombre, comience con un número (liste solo los nombres de los directorios, no su contenido). Ayuda : opción \texttt{-d} de \texttt{ls}.
\item - Cuente con \texttt{wc -l} el listado anterior. ¿Cuántos procesos tiene en ejecución el sistema?.
\item - Ejecute \texttt{ps -ef} , y \texttt{ps -ef | wc -l} . ¿Qué reporta \texttt{ps -ef}?
\item - ¿Coincide la cantidad de procesos reportados con \texttt{ps -ef} con el obtenido en el primer inciso de este ejercicio?
\end{itemize}


\section{Ejercicio 7.}
El comando \texttt{find /var} lista todos los archivos y directorios que existen debajo de \texttt{/var}.
\begin{itemize}
\item - ¿Cuantos archivos y directorios contiene el \texttt{/var} de su sistema?. Ayuda : \texttt{wc}.
\item - ¿Cuanto espacio utiliza en disco todo el \texttt{/var}?. Ayuda: \texttt{du}.
\item - ¿Cuál es el directorio en donde el sistema guarda los mensajes de log?
\item - Mencione 5 archivos de log de su sistema.
\end{itemize}








\end{document}
