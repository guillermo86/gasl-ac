%%% LaTeX Template: Article/Thesis/etc. with colored headings and special fonts
%%%
%%% Source: http://www.howtotex.com/

\documentclass[12pt]{article}


\usepackage{apuntes-estilo}
\usepackage{fancyhdr,lastpage}
\usepackage{verbatim}



\def\maketitle{

% Titulo 
 \makeatletter
 {\color{bl} \centering \huge \sc \textbf{
Trabajo práctico N 4 \\
\large \vspace*{-8pt} \color{black} Arbol de Directorios
 \vspace*{8pt} }\par}
 \makeatother


% Autor
 \makeatletter
 {\centering \small 
	Introducción a la Administración de Sistemas \\
 	Departamento de Ingeniería de Computadoras \\
 	Facultad de Informática - Universidad Nacional del Comahue \\
 	\vspace{20pt} }
 \makeatother

}

% Custom headers and footers
\fancyhf{} % clear all header and footer fields
\fancypagestyle{plain}{\fancyhf{}}
  	\pagestyle{fancy}
 	\lhead{\footnotesize TP N 4 - Administrando Cuentas de Usuario}
 	\rhead{\footnotesize \thepage\ }	% "Page 1 of 2"

\def\ti#1#2{\texttt{#1} & #2 \\ }



\begin{document}

\thispagestyle{empty}
\maketitle
\setlength{\parindent}{0pt}

\paragraph{Los siguientes ejercicios se realizan en su totalidad en el equipo asignado al grupo.
Utilice la redirección y el editor \texttt{vi} para guardar las salidas y editar los resultados 
en un archivo llamado trabajo-practico-4.txt . En el archivo de resolución indique su nombre y apellido, y el número de ejercicio a resolver, junto a la salida incluyendo el comando que ejecuta. }

\section{Ejercicio 1.}


El comando  adduser se utiliza para crear nuevas cuentas de usuario de manera interactiva.
Su utilizacion es sencilla. Solo es necesario, como mínimo, especificar el nombre de usuario. Ejemplo : adduser jgonzalez
El sistema luego, le pedirá, de manera interactiva, información extra de la nueva cuenta a crear (el nombre completo real, la clave inicial, etc).

Utilizando el comando adduser agregue al sistema estos cinco usuarios nuevos :

Nombre de usuario : director
Nombre real : Juan Jose Gonzalez
Clave inicial : director01

Nombre de usuario : secretaria
Nombre real : Susana Gimenez
Clave inicial : secretaria02

Nombre de usuario : contaduria
Nombre real : Contador Esteban Gonzalez
Clave inicial : contador03

Nombre de usuario : legales
Nombre real : Abogado Mirko Garcia 
Clave inicial : legales04

Nombre de usuario : buffet
Nombre real : Mirta Legrant de Gonzalez
Clave inicial : buffet01


\section{Ejercicio 2.}
\begin{itemize}
\item - Utilice \texttt{cat} para visualizar el contenido de \texttt{/etc/passwd}, en la pantalla.
\item - Utilice \texttt{cat} y redirija la salida al archivo en donde coloca los resultados del practico.
\item - ¿Cuántos usuarios hay definidos en el sistema?
Ayuda: Utilice el comando \texttt{cat} redirigiendo la salida a \texttt{wc} (word count) para contar cuantas lineas
existen en el archivo.
\end{itemize}


\section{Ejercicio 3.}
Otra manera muy útil de verificar usuarios es a través de sesiones de login al sistema.

Verifique los cinco usuarios creados realizando conexiones secure shell (ssh) al equipo asignado al grupo.
Utilice el nombre de usuario y su clave inicial para corroborar que puede ingresar al sistema.

¿Cuál es el directorio HOME de cada usuario?



\section{Ejercicio 4.}
Realice una conexion al sistema con el usuario director.
Y responda :

¿Cuál es el directorio HOME del usuario director? ¿Que variable del ambiente del usuario reporta esa información?
¿Con qué comando verifica todas las variables del ambiente del usuario director?

Intente crear un nuevo usuario utilizando el usuario director.
¿Es posible crear un nuevo usuario con director? Justifique porqué.



\section{Ejercicio 4.}
El comando addgroup funciona de manera similar a adduser, pero se utiliza para crear nuevos grupos de usuario.
Ejemplo de uso : addgroup empleados (crea en el sistema un grupo llamado "empleados").


Utilice addgroup para crear tres grupos nuevos en el sistema:

\begin{itemize}
\itme directorio
\itme empleados
\itme todos
\end{itemize}

Además, puede utilizar el comando usermod para agregar usuarios a grupos, o modificar algunas opciones del usuario.
Por ejemplo, para agregar el usuario "director" al grupo "todos", puede ejecutar "usermod -G todos -a director"

Modifique los grupos utilizando usermod para agregar usuarios a los grupos. 
El grupo directorio deberia estar compuesto por los usuario director, legales, contaduria.
El grupo empleados deberia estar compuesto por secretaria, contaduria, legales, buffet.
El grupo todos debería estar compuesto por director, legales, contaduria.

Utilice el comando vigr para editar los grupos con vi. Agregue manualmente, al grupo "todos" a los usuarios buffet y secretaria.


Quitar con el comando deluser a los usuarios buffet y legales.
Observe el archivo /etc/passwd para verificar que los usuarios han sido creados.


Listar los usuarios del sistema con cat. 
Utilice los comando cat y "wc" para obtener la cantidad de usuarios del sistema.
Utilice los comando cat y "sort" para generar el mismo listado, pero en orden alfabetico.

Repaso sistema de archivos.

Listar los directorios /bin y /usr/bin por pantalla. ¿Qué tipo de archivos existen en esos directorios? (Ejemplo: Binarios, ejecutables, de texto plano, de configuracion, variables, temporarios, de documentacion. Un directorio puede tener mas de un tipo mencionado, por ejemplo "de texto y de configuracion").
Listar los directorios /etc por pantalla. Visualizar con cat de algun archivo a eleccion. ¿Qué tipo de archivos existen en esos directorios?
Listar el directorio /home. ¿Que tipo de información se guarda en ese directorio?
Listar el directorio /tmp. ¿Para qué sirve ese directorio?
Ejecutar find /var, y utilizar en conjunción con wc para conocer la cantidad de archivos que existe en el directorio /var.



Qué tipo de archivos se guarda en 
Utilice el usuario 
Utilice el comando cat para verifiel archivo passwd pque han sido eliminados.
Verifi

\begin{enumerate}
\item Sobre el directorio \texttt{/etc}
\begin{itemize}
\item - ¿Qué tipo de archivos hay en ese directorio?
\end{itemize}

\item El archivo \texttt{/etc/passwd} contiene el listado de usuarios del sistema.
Cada linea de texto en ese archivo define los detalles de un único usuario.

\begin{itemize}
\item - Utilice \texttt{cat} para visualizar el contenido de \texttt{/etc/passwd}, en la pantalla.
\item - Utilice \texttt{cat} y redirija la salida al archivo en donde coloca los resultados del practico.
\item - ¿Cuántos usuarios hay definidos en el sistema?
Ayuda: Utilice el comando \texttt{cat} redirigiendo la salida a \texttt{wc} (word count) para contar cuantas lineas
existen en el archivo.
\end{itemize}


\item El archivo \texttt{/etc/shadow} contiene las passwords de los usuarios de manera encriptada.
\begin{itemize}
\item - ¿Que permisos tiene ese archivo? Utilice el comando ls y redirija la salida a trabajo-practico-3.txt
\item - ¿Quien puede leer y modificar ese archivo?
\item - ¿Puede un usuario común modificar \texttt{/etc/shadow}? ¿Explique porqué?
\end{itemize}


\item El archivo \texttt{/etc/debian\_version} contiene en texto plano el nombre de la
versión de Debian Linux instalada.
\begin{itemize}
\item - ¿Qué versión de Debian utiliza en la práctica?
\end{itemize}
\end{enumerate}


\section{Ejercicio 2.}
Sobre el directorio \texttt{/usr/bin}

\begin{itemize}
\item - ¿Qué tipo de archivos hay en ese directorio?
\item - ¿Cuáles son los permisos del directorio \texttt{/usr/bin}?
\item - ¿Quien puede modificar el contenido de ese directorio?
\item - ¿Por qué el directorio debe tener esos permisos, y no otros?
\end{itemize}


\section{Ejercicio 3.}
El comando \texttt{du} se puede utilizar para ver el espacio utilizado por archivos y directorios.
Si ejecuta \texttt{du -h /etc} el sistema muestra de manera legible (el argumento \texttt{-h} significa "legible para las personas")
el espacio utilizado en el directorio \texttt{/etc} (suma los tamaños de todos los archivos y directorios que estén dentro de \texttt{/etc}).

\begin{itemize}
\item - ¿Cuanto espacio utiliza el directorio \texttt{/usr}?
\item - ¿Porqué el directorio \texttt{/usr} ocupa mas espacio que \texttt{/etc}?
\item - ¿Cuanto espacio utiliza el directorio raíz \texttt{/} ?
\end{itemize}


\section{Ejercicio 4.}
\begin{enumerate}
\item El comando \texttt{hostname} se utiliza para conocer el nombre del sistema (o para definirlo).
\begin{itemize}
\item - ¿Qué nombre tiene su sistema?
\end{itemize}
\end{enumerate}

\begin{enumerate}
\item El programa \texttt{hostname} viene empaquetado en un paquete de nombre homónimo.
El comando \texttt{dpkg -L hostname}  lista los archivos que el paquete \texttt{hostname} instaló en el sistema.
\begin{itemize}
\item - Indique que son cada uno de los archivos listados (por ejemplo: si son binarios ejecutables, archivos de configuración, archivos de log, etc).
\end{itemize}
\end{enumerate}



\section{Ejercicio 5.}
El archivo \texttt{/proc/meminfo} contiene información de la memoria utilizada en kilobytes.

\begin{itemize}
\item - Utilice cat para ver la información.
\item - ¿Cuanta memoria total en MegaBytes tiene el sistema?
\item - ¿Cuanta memoria se está utilizando del total? (en MegaBytes). Ayuda: utilice el dato MemFree
\end{itemize}


\section{Ejercicio 6.}
En el directorio \texttt{/proc} existen directorios con nombres "numéricos", que corresponden con el PID de los procesos del sistema.
Estos directorios contienen información de los procesos (de los programas en ejecución).
\begin{itemize}
\item - Liste los directorios en \texttt{/proc}, en el cual, su nombre, comience con un número (liste solo los nombres de los directorios, no su contenido). Ayuda : opción \texttt{-d} de \texttt{ls}.
\item - Cuente con \texttt{wc -l} el listado anterior. ¿Cuántos procesos tiene en ejecución el sistema?.
\item - Ejecute \texttt{ps -ef} , y \texttt{ps -ef | wc -l} . ¿Qué reporta \texttt{ps -ef}?
\item - ¿Coincide la cantidad de procesos reportados con \texttt{ps -ef} con el obtenido en el primer inciso de este ejercicio?
\end{itemize}


\section{Ejercicio 7.}
El comando \texttt{find /var} lista todos los archivos y directorios que existen debajo de \texttt{/var}.
\begin{itemize}
\item - ¿Cuantos archivos y directorios contiene el \texttt{/var} de su sistema?. Ayuda : \texttt{wc}.
\item - ¿Cuanto espacio utiliza en disco todo el \texttt{/var}?. Ayuda: \texttt{du}.
\item - ¿Cuál es el directorio en donde el sistema guarda los mensajes de log?
\item - Mencione 5 archivos de log de su sistema.
\end{itemize}








\end{document}
