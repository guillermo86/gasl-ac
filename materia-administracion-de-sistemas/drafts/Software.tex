%%% LaTeX Template: Article/Thesis/etc. with colored headings and special fonts
%%%
%%% Source: http://www.howtotex.com/

\documentclass[12pt]{article}


\usepackage{apuntes-estilo}
\usepackage{fancyhdr,lastpage}
\usepackage{color,colortbl}
\usepackage{verbatim}

\def\maketitle{

% Titulo 
 \makeatletter
 {\color{bl} \centering \huge \sc \textbf{
  Instalación, configuración y actualización de software.\\ 
\large \vspace*{-8pt} \color{black} Guía básica de administración de procesos. 
 \vspace*{8pt} }\par}
 \makeatother

% Autor
\makeatletter
 {\centering \small 
 	Departamento de Ingeniería de Computadoras \\
 	Facultad de Informática - Universidad Nacional del Comahue \\
 	\vspace{20pt} }
 \makeatother

}

% Custom headers and footers
\fancyhf{} % clear all header and footer fields
\fancypagestyle{plain}{\fancyhf{}}
  	\pagestyle{fancy}
 	\lhead{\footnotesize Administración de software - Departamento de Ingeniería de Computadoras}
 	\rhead{\footnotesize \thepage\ }	% ''Page 1 of 2''

\def\ti#1#2{\texttt{#1} & #2 \\ }



\begin{document}

\thispagestyle{empty}
\maketitle
\setlength{\parindent}{0pt}

\section*{Introducción}

Si bien la instalación, actualización y remoción de software hoy en día
puede parecer trivial, su mala administración suele ser una fuente de 
errores complejos que pueden derivar en daños graves al sistema. 
En particular en los sistemas de tipo UNIX como GNU/Linux el ecosistema
de aplicaciones es delicado, y para nada anárquico. Dicho ecosistema tiene 
una estructura y una forma particular de manipularlo. 

UNIX y variantes de sistemas GNU/Linux utilizan, generalmente, algún
sistema de empaquetado para facilitar la tarea de la administración de software.
Los {\bf paquetes} se han utilizado tradicionalmente para distribuir programas,
pero pueden ser también utilizados para contener archivos de configuración
o datos compartidos utilizados por varios paquetes. Su objetivo principal
es tratar de lograr un proceso de instalación tan atómico como sea posible,
de manera que si ocurre un error en el proceso, el administrador
sólo debe pausar la operación en curso (instalación, desistalación, etc)
y el sistema no quede en algún estado inconsistente.

Existen diversas maneras de gestionar el software en una distribución 
GNU/Linux. La forma más común y recomendada para un administrador es a través de paquetes
provistos por la organización que desarrolla la distribución GNU/Linux.
Pero también existen otros mecanismos
para la instalación de software en sistemas GNU/Linux :

- En forma de código fuente, en donde se debe compilar e instalar mediante herramientas de desarrollo
- En forma binaria provista por terceras partes
- En forma de paquetes provisto por terceras partes

Estos mecanismos y otros no listados quedan fuera del alcance de este artículo.

\section*{Administrando Paquetes de Software}

Las tareás básicas en la Administración de Software son :
\begin{itemize}
\item Actualizar la lista de paquetes disponibles;
\item Instalar, reinstalar, actualizar, y eliminar paquetes de software;
\item Obtener información acerca de los paquetes, incluyendo la versión, estado, dependencias, tamaño, integridad, etc;
\item Determinar qué archivos proporciona el paquete, y descubrir cual de los paquetes contiene un archivo determinado.
\end{itemize}

Existen una serie de conceptos que la 
mayoría de las distribuciones implementan y que un administrador de 
sistemas debe conocer con certeza aún cuando no conozca la implementación
particular de cada distribución.  


\subsection*{Paquete de software}


Un paquete de software es una serie de programas que se distribuyen conjuntamente. Algunas de las razones suelen ser que el funcionamiento de cada uno complementa a o requiere de otros. Un paquete es un único archivo.

Los paquetes incluyen información importante, además del software mismo, como pueden ser el nombre completo, una descripción de su funcionalidad, el número de versión, el distribuidor del software, la suma de verificación y una lista de otros paquetes requeridos para el correcto funcionamiento del software. Esta metainformación se introduce normalmente en una base de datos de paquetes local.

Existen diferentes formatos de paquetes utilizados, pero 
dos formatos de paquetes binarios ampliamente difundidos son los formatos DEB y RPM.

{\bf deb} es el formato y la extensión del nombre de los archivos de paquetes de software de la distribución de Linux Debian y derivadas (e.j.Ubuntu), y, el nombre más usado para dichos paquetes. 
{\bf rpm} es el formato y extensión del nombre de los archivos de paquetes de la distribucion Red Hat y muchas otras.


\subsection*{Repositorio de Paquetes de software}


Un repositorio es un directorio o sitio web que contiene paquetes de software y archivos de índices. 
Las utilidades de administración de software como yum o apt (definiremos yum y apt mas adelante)
automáticamente ubican y obtienen 
los paquetes desde esos repositorios. Este método libera al administrador de tener que buscar e instalar las nuevas aplicaciones o actualizaciones de forma manual. 

Generalmente, las distribuciones GNU/Linux proporcionan una red de servidores que proveen 
el servicio de repositorios de software a los usuarios, de modo que siempre
esté disponible el servicio para la instalación de programas en estos sistemas.

ATENCION: Los desarrolladores de software de terceros también proporcionan repositorios
para sus paquetes compatibles con distintas distribuciones GNU/Linux. Pero,
es importante recordar que {\bf no es recomendado instalar software que no se encuentre en los 
repositorios oficiales}. Además, es frecuente que al instalar software de terceras 
partes el administrador deba utilizar métodos manuales.

\subsection*{Sistema de Gestión de Paquetes de Software}

Un sistema de gestión de paquetes, también conocido como gestor de paquetes, es una colección de herramientas que sirven para automatizar el proceso de instalación, actualización, configuración y eliminación de paquetes de software.

Existen decenas de sistemas de gestión de paquetes. Cada distribución GNU/Linux opta 
por un sistema oficialmente. Dos de los sistemas mas utilizados en distribuciones
que utilizan paquetes binarios son dpkg y rpm.

{\bf dpkg} es la base del sistema de gestión de paquetes creado por el proyecto Debian GNU/Linux, mientras que {\bf rpm}
fue desarrollado por Red Hat.
Ambos sistemas se utilizan en las principales distribuciones GNU/Linux. dpkg y el formato
deb se utilizan en Debian, Ubuntu, Tuquito, Mint, y muchas otras. RPM es utilizado por
las distribuciones Fedora, CentOS, Red Hat, Suse Linux, y varias mas.



\section*{Licencia}

Este texto fue creado por Miriam Tamara Lechner y se encuentra bajo 
Licencia Creative Commons Atribución-CompartirDerivadasIgual 3.0 Unported

\end{document}
