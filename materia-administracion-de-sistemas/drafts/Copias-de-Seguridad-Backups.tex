\documentclass[12pt]{article}


\usepackage{apuntes-estilo}
\usepackage{fancyhdr,lastpage}
\usepackage{color,colortbl}
\usepackage{verbatim}

\def\maketitle{

% Titulo 
 \makeatletter
 {\color{bl} \centering \huge \sc \textbf{
 Copias de respaldo (backup)\\ 
\large \vspace*{-8pt} \color{black} Guía básica de copias de respaldo (backup)
 \vspace*{8pt} }\par}
 \makeatother


% Autor
 \makeatletter
 {\centering \small 
 	Departamento de Ingeniería de Computadoras \\
 	Facultad de Informática - Universidad Nacional del Comahue \\
 	\vspace{20pt} }
 \makeatother

}

% Custom headers and footers
\fancyhf{} % clear all header and footer fields
\fancypagestyle{plain}{\fancyhf{}}
  	\pagestyle{fancy}
 	\lhead{\footnotesize Copia de respaldo - Departamento de Ingeniería de Computadoras}
 	\rhead{\footnotesize \thepage\ }	% ''Page 1 of 2''

\def\ti#1#2{\texttt{#1} & #2 \\ }



\begin{document}

\thispagestyle{empty}
\maketitle
\setlength{\parindent}{0pt}


\section{Importancia de las copias de seguridad}
 En este apunte se explica cuándo, cómo y  por qué hacer copias de 
seguridad (backups), y de cómo recuperar información de las 
copias realizadas.

Sus datos son valiosos. Tomara tiempo y esfuerzo, si fuese necesario,
re-crearlos, y esto cuesta dinero o al menos esfuerzo extra del personal.
Algunas veces los datos no pueden ser re-creados, si por ejemplo son el 
resultado de algunos experimentos. Debido a que los datos elaborados son una 
inversión, debe protegerlos y tomar medidas para evitar pérdidas.

Existen al menos cuatro razones por la que puede perder datos: 
\begin{itemize}
\item Fallas de hardware.
\item Errores en el software. 
\item Acción humana (accidental o intencional). 
\item Desastres (incendios, inundaciones, etc). 
\end{itemize}

 Si bien el hardware moderno tiende a ser confiable, puede llegar a dañarse
aparentemente de manera espontánea. La pieza más crítica para 
almacenar datos es el disco rígido, ya que se encuentra compuesto de 
pequeñísimos campos magnéticos que deben mantenerse intactos en un mundo 
lleno de interferencias electromagnéticas.  
Por otra parte el software moderno tiende a ser no confiable; un programa 
sólido como una roca es una excepción, no una regla. Como si esto fuera 
poco, las personas son completamente no confiables, suelen confundirse o 
equivocarse, o pueden ser maliciosos y destruir los datos de forma 
intencional. La naturaleza no puede ser malvada, pero podría llegar a 
realizar estragos. 

Resumiendo: en computación, es un pequeño milagro que algo trabaje del todo bien.

	

Las copias de seguridad son una manera de proteger la inversión realizada en 
los datos. Las pérdidas de información no es tan importante si existen varias
copias resguardadas (existe solo el costo que conlleve recuperar los datos 
perdidos desde las copias).

	

Es importante realizar copias de seguridad correctamente. Como todo lo 
relacionado con el mundo físico, se dañarán tarde o temprano. Parte del
trabajo
al realizar copias de seguridad es estar seguro de que estas funcionan; 
ya que no desea 
enterarse tiempo después que las copias no son útiles.\footnote{No reírse. Esto le sucede a muchas personas.
} Además, piense
en estos dos casos: sus datos podrían dañarse justo en el momento en que
esta realizando copias de respaldo; o, si solamente tiene un medio para
copias de seguridad, se podría llegar a romper también, 
dejándolo solo con las cenizas de todo lo fumado mientras realizaba el 
trabajo duro.
	
		\footnote{Been there, done that...}
		

O se entera, cuando intenta recuperar, 
que olvidó'o respaldar algo importante, como la 
base de datos de los usuarios en un sitio con 15000. Finalmente, el mejor 
de todos los casos:
las copias de seguridad trabajan perfectamente, pero la 
última unidad sobre la faz de la Tierra que lee el tipo de cinta que 
usted utilizaba, está llena de agua y se ha dañado irreparablemente.

	 

Cuando de copias de seguridad se trata, la paranoia está en la descripción de 
la tarea.




\section{

Seleccionando el medio de backup}

	

La decisión mas importante al pensar en hacer copias de seguridad es la 
selección del medio a utilizar. Necesita considerar el costo, confiabilidad, velocidad, 
disponibilidad y usabilidad.

	

El costo es importante, porque preferentemente desea contar con mayor capacidad
de almacenamiento para los backups de lo que necesita para los datos existentes.
Un medio barato es usualmente casi una obligación.

	

La confiabilidad es un ítem de extrema importancia, ya que una copia de respaldo 
dañada puede hacer llorar a un gigante. Un medio para copias de seguridad 
debe ser capaz de mantener los datos en perfecto estado durante años. 
Además, la forma en que se utiliza el medio afecta a su confiabilidad.
Un disco rígido es típicamente muy confiable, pero no como 
medio para copias de seguridad en caso de que se encuentre en la misma computadora 
en donde están los disco al que se les realiza la copia.

	

La velocidad usualmente no muy importante, 
en caso de que la copia pueda ser
realizada sin interacción. No importa que el backup demore unas 
dos horas, o lo que fuese necesario si no necesita atención. En cambio, si 
la copia no puede ser realizada cuando la computadora se encuentre ociosa, 
considere a la velocidad del medio al momento de la elección.

	

La disponibilidad es obviamente necesaria, debido a que no se puede utilizar 
un medio para copias de seguridad si no existe. Menos obvio es la 
disponibilidad futura, y en otras
computadoras diferentes a las que se utilizaron para generar las copias.
Si este caso sucede, puede no ser posible realizar una recuperación de la 
información luego de un desastre.

	

La practicidad es un gran factor que se relaciona con la frecuencia en que
las copias son realizadas. 
Cuanto mas fácil de usar sea el medio para realizar las copias, mejor.
Un medio 
para copias de seguridad no debe ser difícil, o aburrido de 
utilizar.

	

Las alternativas típicas son los discos flexibles y las cintas. Los discos 
flexibles son muy baratos, relativamente confiables, no muy rápidos, muy 
disponibles, pero no muy útiles para grandes cantidades de información. Las 
cintas varían en cuanto a su valor, generalmente son baratas aunque algunos tipos
no lo son tanto,
relativamente confiables y veloces, muy disponibles, y generalmente (aunque depende de 
su capacidad) son útiles para almacenar mucha información.

	

Existen otras alternativas. Usualmente no son muy comunes,
pero si la disponibilidad no es un problema, pueden ser una mejor opción en muchos 
casos. Por ejemplo, los discos magneto-ópticos, ya que pueden tener las ventajas 
de los discos flexibles (acceso aleatorio, rápida recuperación de un único archivo)
y las ventajas de las cintas (gran capacidad de almacenamiento).



\section{

Seleccionando la herramienta de backup}

	

Existen muchas herramientas que pueden ser utilizadas para realizar las copias 
de seguridad. Las herramientas tradicionales en entornos UNIX 
son \texttt{\textbf{tar}}, \texttt{\textbf{cpio}}, y
	\texttt{\textbf{dump}}.. Además, existe un gran número de paquetes de terceros 
(comerciales y libres) que pueden ser utilizados. La selección del medio para 
copias de seguridad puede afectar a la selección de la herramienta a utilizar.

	 \texttt{\textbf{tar}} y \texttt{\textbf{cpio}}
son similares, y casi completamente equivalentes desde el punto de 
vista de los backups. Ambas son capaces de almacenar y recuperar 
archivos en cintas. También son capaces de utilizar 
prácticamente cualquier medio, debido a que los controladores de dispositivos del kernel son 
los que se encargan del acceso al hardware a bajo nivel; por lo que todos los dispositivos
tienden a verse de la misma manera para los programas en espacio de usuario.
Algunas versiones UNIX de \texttt{\textbf{tar}}
	y \texttt{\textbf{cpio}}
pueden tener dificultades con archivos inusuales 
(enlaces simbólicos, archivos de dispositivos, archivos con nombres
muy largos, etc.), pero las versiones GNU/Linux de estas herramientas deberían 
manejar toda clase de archivos correctamente.

	 \texttt{\textbf{dump}}
es diferente a las dos herramientas anteriores, debido a que lee el 
sistema de archivos directamente, y no a través del sistema de archivos. 
Además, fue desarrollado específicamente para generar copias de seguridad;
 \texttt{\textbf{tar}}
	y \texttt{\textbf{cpio}} son empaquetadores de archivos (a pesar de que también trabajan como 
herramientas para backups).

	

Leer el sistema de archivos directamente tiene algunas ventajas. Es posible 
realizar copias sin afectar las marcas de tiempo de los archivos; 
en cambio, para \texttt{\textbf{tar}} y \texttt{\textbf{cpio}}, es necesario primero montar el 
sistema de archivos con permisos de solo-lectura. Si es necesario copiar 
todo el sistema de archivos, entonces la lectura directa también es mas efectiva,
debido a que se realizan 
muchos menos movimientos de la cabeza lecto-escritora del disco. La mayor desventaja
es que dump es un programa de copia específico para solamente un tipo de sistemas
de archivos;
el programa \texttt{\textbf{dump}}de GNU/Linux puede leer únicamente el sistema
de archivos ext2.

	 \texttt{\textbf{dump}}
también soporta distintos niveles de copias (tema que se encuentra
explicado en páginas posteriores)
; con \texttt{\textbf{tar}} y \texttt{\textbf{cpio}} los niveles de copia de respaldo debe 
ser implementado utilizando otras herramientas.

	

Una comparación con herramientas de terceros para copias de seguridad se 
encuentra fuera del alcance de este libro. El Mapa de Software para GNU/Linux 
lista muchos de ellos que son gratuitos.



\section{

Copias de respaldo simples}

	

Un esquema simple de copias de seguridad es respaldar todo una única vez, y 
luego, copiar solamente los archivos que fueron modificados después de la copia inicial.
La primera copia se denomina \textit{copia total (full
	backup)}, y las 
siguientes son \textit{copias incrementales (incremental
	backups)}. Una copia total frecuentemente
necesita de mayor esfuerzo para ser generada, debido a que hay mas datos a escribir,
y eventualmente (o frecuentemente []) puede no caber en una única cinta, o cualquiera
que sea el medio utilizado.
De manera opuesta, recuperar información desde las copias
incrementales necesita muchas veces mas empeño que hacerlo desde una copia 
total. La recuperación puede ser optimizada si cada copia incremental se realiza 
con respecto a la copia total previa. De esta manera, las copias pueden 
necesitar un poquito mas de esfuerzo (y demorar mas también), pero nunca será
necesario recuperar mas que dos copias (una total y una incremental).

	

En caso de que desee realizar copias todos los días y tenga seis cintas, puede utilizar 
la cinta 1 para la primera copia completa (digamos, un Viernes), y utiliza
las cintas 2 a 5 para las copias incrementales (Lunes a Jueves). El segundo Viernes,
realiza una nueva copia total en la cinta 6, y reinicia nuevamente el ciclo de copias incrementales
con las cintas 2 a 5. No es conveniente sobreescribir la cinta 1 hasta que una nueva
copia completa haya sido generada. Después del segundo Viernes, debe mantener la cinta 1
en otro lugar distinto al del resto. De esta manera, si el conjunto de cintas 2 a 6 
se dañan en un incendio, tiene al menos una copia completa en un lugar seguro (esta
copia tiene una semana de antigüedad, pero es mejor que nada).
Cuando 
necesite realizar la próxima copia total, utilice la cinta 1 y coloque la cinta 6
el el lugar seguro.

	

En caso de que disponga con mas de 6 cintas, puede utilizar las cintas extras
para las copias completas. Y cada vez que realice una copia total utilizará la 
cinta mas antigua. De esta manera puede almacenar copias
completas de varias semanas previas, lo cual es útil en caso de que necesite
encontrar un archivo que ha sido borrado, o desea recuperar una vieja versión de
algún otro.

\subsection{

Realizando copias de seguridad con tar}

	

Una copia completa puede realizarse fácilmente con \texttt{\textbf{tar}}:


<prompt>#</prompt> <userinput>tar --create --file /dev/ftape 
/usr/src</userinput>
\colorbox{grey}{\parbox[t]{0.95\linewidth}{ \vspace*{0.5cm} {\tt tar: Removing leading / from absolute path names in 
the archive } \vspace*{0.5cm} } } 
<prompt>#</prompt>


El ejemplo anterior utiliza la versión GNU de \texttt{\textbf{tar}} y sus nombres de opciones 
largos. La versión tradicional de \texttt{\textbf{tar}} solo comprende opciones de un único caracter.
Además, la versión GNU puede manejar 
copias que no quepan completamente en una cinta (o medio utilizado), y con caminos de
directorios muy largos; no todas las versiones de tar pueden hacer eso. 
(GNU/Linux solo utiliza GNU \texttt{\textbf{tar}}.)  

En caso de que la copia no quepa en una única cinta, es necesario activar la opción
<option>--multi-volume</option> (<option>-M</option>):


<prompt>#</prompt> <userinput>tar -cMf /dev/fd0H1440 
/usr/src</userinput>
\colorbox{grey}{\parbox[t]{0.95\linewidth}{ \vspace*{0.5cm} {\tt tar: Removing leading / from absolute path names in 
the archive
Prepare volume #2 for /dev/fd0H1440 and hit return: } \vspace*{0.5cm} } } 
<prompt>#</prompt>



Observe que debe dar formato a todos los disquetes a utilizar antes de comenzar 
la copia de respaldo, o utilice otra ventana o terminal virtual y dele 
formato al disquete cuando \texttt{\textbf{tar}} le solicite uno nuevo.

	

Después de realizar una copia, debe verificar si todo se encuentra en correctas condiciones
utilizando la opción <option>--compare</option> (<option>-d</option>):


<prompt>#</prompt> <userinput>tar --compare --verbose -f 
/dev/ftape</userinput>
\colorbox{grey}{\parbox[t]{0.95\linewidth}{ \vspace*{0.5cm} {\tt usr/src/
usr/src/linux
usr/src/linux-1.2.10-includes/
.... } \vspace*{0.5cm} } } 
<prompt>#</prompt>



Si no verifica las copias, entonces no percibirá que las copias de respaldo no 
son útiles hasta que estas sean necesarias.
	
	

Una copia de respaldo incremental puede ser realizada utilizando la opción 
<option>--newer</option>
	(<option>-N</option>) de \texttt{\textbf{tar}}:


<prompt>#</prompt> <userinput>tar --create --newer '8 Sep 1995' 
--file /dev/ftape /usr/src 
--verbose</userinput>
\colorbox{grey}{\parbox[t]{0.95\linewidth}{ \vspace*{0.5cm} {\tt tar: Removing leading / from absolute path names in 
the archive
usr/src/
usr/src/linux-1.2.10-includes/
usr/src/linux-1.2.10-includes/include/
usr/src/linux-1.2.10-includes/include/linux/
usr/src/linux-1.2.10-includes/include/linux/modules/
usr/src/linux-1.2.10-includes/include/asm-generic/
usr/src/linux-1.2.10-includes/include/asm-i386/
usr/src/linux-1.2.10-includes/include/asm-mips/
usr/src/linux-1.2.10-includes/include/asm-alpha/
usr/src/linux-1.2.10-includes/include/asm-m68k/
usr/src/linux-1.2.10-includes/include/asm-sparc/
usr/src/patch-1.2.11.gz } \vspace*{0.5cm} } } 
<prompt>#</prompt>



Desafortunadamente, \texttt{\textbf{tar}} no puede conocer cuando la información en los inodos
de los archivos
ha cambiado, como por ejemplo, si sus permisos o nombre ha sido modificado. 
Puede solucionar este inconveniente utilizando \texttt{\textbf{find}}, y comparar el estado del sistema de 
archivos actual con una lista de archivos que fueron respaldados previamente. Los 
scripts y programas que realizan esta tarea pueden ser encontrados en los sitios ftp de 
GNU/Linux.
	


\subsection{

Recuperando archivos con \texttt{\textbf{tar}}}

	

Debe utilizar la opción  <option>--extract</option> (<option>-x</option>) para recuperar archivos que fueron 
previamente respaldados con \texttt{\textbf{tar}}:


<prompt>#</prompt> <userinput>tar --extract --same-permissions 
--verbose --file 
/dev/fd0H1440</userinput>
\colorbox{grey}{\parbox[t]{0.95\linewidth}{ \vspace*{0.5cm} {\tt usr/src/
usr/src/linux
usr/src/linux-1.2.10-includes/
usr/src/linux-1.2.10-includes/include/
usr/src/linux-1.2.10-includes/include/linux/
usr/src/linux-1.2.10-includes/include/linux/hdreg.h
usr/src/linux-1.2.10-includes/include/linux/kernel.h
... } \vspace*{0.5cm} } } 
<prompt>#</prompt>



También puede extraer archivos y directorios específicos (los cuales incluyen 
todos sus archivos y subdirectorios). Basta con mencionarlos en la línea de comandos:


<prompt>#</prompt> <userinput>tar xpvf /dev/fd0H1440 
usr/src/linux-1.2.10-includes/include/linux/hdreg.h</userinput>
\colorbox{grey}{\parbox[t]{0.95\linewidth}{ \vspace*{0.5cm} {\tt usr/src/linux-1.2.10-includes/include/linux/hdreg.h } \vspace*{0.5cm} } } 
<prompt>#</prompt>



Utilice la opción <option>--list</option> (<option>-t</option>) para conocer cuales son los archivos presentes en un 
volumen de backup:


<prompt>#</prompt> <userinput>tar --list --file 
/dev/fd0H1440</userinput>
\colorbox{grey}{\parbox[t]{0.95\linewidth}{ \vspace*{0.5cm} {\tt usr/src/
usr/src/linux
usr/src/linux-1.2.10-includes/
usr/src/linux-1.2.10-includes/include/
usr/src/linux-1.2.10-includes/include/linux/
usr/src/linux-1.2.10-includes/include/linux/hdreg.h
usr/src/linux-1.2.10-includes/include/linux/kernel.h
... } \vspace*{0.5cm} } } 
<prompt>#</prompt>



Note que \texttt{\textbf{tar}} siempre lee el volumen de la copia de respaldo secuencialmente, 
lo que puede llegar a ser lento para grandes volúmenes. 
De cualquier manera, si utiliza medios secuenciales como cintas, no 
es posible el acceso aleatorio.
	
	 \texttt{\textbf{tar}}
no maneja correctamente archivos eliminados. En caso de que se necesite recuperar un 
sistema de archivos desde una copia completa y otra incremental, y se han 
eliminado archivos entre las dos copias, estos archivos existirán luego de
la recuperación. Este tipo de casos puede llegar a ser un gran problema si los archivos 
tienen datos sensibles que no deberían existir más.





\section{

Copias de seguridad de múltiples niveles}

	

Las copias de respaldo simples que están explicadas en las secciones previas son 
frecuentemente adecuadas para uso personal o de pequeñas corporaciones. Para
tareas mas complejas puede ser apropiado el uso de copias de seguridad
de múltiples niveles.

	

El método simple tiene dos niveles de copias: total e incremental. Es
posible generalizar este método para cualquier número de niveles. Una copia completa
sería el nivel 0, y los niveles diferentes de copias incrementales 1, 2, 
3, etc. Por cada nivel de copia incremental se copia todo lo que haya sido 
modificado desde la copia previa del mismo nivel o de uno menor.

	

El propósito de este procedimiento es mantener un \textit{histórico de copias (backup history)}
amplio de manera económica.
En el ejemplo de la sección previa, el histórico del backup
se remonta a la copia de respaldo completa previa (una semana).
Si agrega mas cintas este tiempo puede ser extendido, aunque 
el costo (monetario) es bastante alto, y solo puede ampliar 
el histórico en una semana por cada cinta nueva.
Aún así, un histórico mas amplio es
muy útil, ya que los archivos que se dañan o son borrados pueden pasar desapercibidos
por un largo período de tiempo.
Por lo que recuperar al menos una versión no muy 
actual de un archivo es mejor que nada.

	


El histórico de copias de seguridad multiniveles puede ser extendido de una 
forma más económica.
Por ejemplo, en caso de que contemos con 10 cintas disponibles, podemos 
utilizar las cintas 1 y 2 para las copias mensuales (primer Viernes de cada mes), 
las cintas 3 a 6 para las copias semanales (otros Viernes del mes; note que 
utilizamos 4 cintas debido a que pueden existir cinco Viernes en un mes),
y las cintas 7 a 10 para las copias diarias (Lunes a Jueves). Con solo 
cuatro cintas más que el caso anterior, estamos en condiciones de extender 
de dos semanas a dos meses el histórico de los backups.
Es cierto que no podemos recuperar todas las 
versiones de cada archivo durante esos dos meses, pero 
que sea posible recuperarlo es normalmente suficiente.

	La <xref linkend="backup-history-timeline"/>

muestra el nivel de backup que es utilizado cada día, 
y desde cuales copias se pueden recuperar archivos.

		<figure id="backup-history-timeline" float="1">
		A sample multilevel backup schedule.}
		<graphic fileref="backup-timeline.png"/>
		</figure>

	

Los niveles de backups pueden también ser utilizados para minimizar el tiempo de
restauración de los sistemas de archivos.
En caso de que existan muchas copias incrementales con números de niveles
de crecimiento muy grande, es muy probable que necesite recuperar de todas 
ellas para reconstruir el sistema de archivos entero.
Sin embargo, puede utilizar números de niveles que no sean monótonos, y mantener
bajo el número de copias que debe utilizar en cada restauración.

	

Para minimizar el número de cintas necesarias en cada recuperación, puede utilizar un 
un número de nivel mas chico por cada cinta incremental.
Sin embargo, el tiempo necesario para realizar las copias crece (en cada backup se debe
copiar todo lo que ha sido modificado desde la última copia total).
Existe un mejor esquema sugerido por
la página de manual de \texttt{\textbf{dump}}, el cual es presentado en la tabla
(niveles-de-copias-de-respaldo-eficientes). Utilizar la siguiente sucesión de 
niveles de copias de seguridad: 3, 2, 5, 4, 7, 6, 9, 8, 9, etc. Este esquema mantiene
bajo los tiempos de copia y de recuperación, y note que lo máximo que se copia son 
dos días.
El número de cintas a ser restauradas depende del intervalo entre las copias completas,
pero ciertamente será menor que el de los esquemas mas simples.

<table id="efficient-backup-levels">
Efficient backup scheme using many backup levels}
<tgroup cols="4">
<thead>
<row><entry>Tape</entry> <entry>Level</entry> <entry>Backup 
(days)</entry> <entry>Restore 
tapes</entry></row>
</thead>
<tbody>
<row><entry>1</entry> <entry>0</entry> <entry>n/a</entry> 
<entry>1</entry></row>
<row><entry>2</entry> <entry>3</entry> <entry>1</entry> <entry>1, 
2</entry></row>
<row><entry>3</entry> <entry>2</entry> <entry>2</entry> <entry>1, 
3</entry></row>
<row><entry>4</entry> <entry>5</entry> <entry>1</entry> <entry>1, 2, 
4</entry></row>
<row><entry>5</entry> <entry>4</entry> <entry>2</entry> <entry>1, 2, 
5</entry></row>
<row><entry>6</entry> <entry>7</entry> <entry>1</entry> <entry>1, 2, 
5, 6</entry></row>
<row><entry>7</entry> <entry>6</entry> <entry>2</entry> <entry>1, 2, 
5, 7</entry></row>
<row><entry>8</entry> <entry>9</entry> <entry>1</entry> <entry>1, 2, 
5, 7, 8</entry></row>
<row><entry>9</entry> <entry>8</entry> <entry>2</entry> <entry>1, 2, 
5, 7, 9</entry></row>
<row><entry>10</entry> <entry>9</entry> <entry>1</entry> <entry>1, 2, 
5, 7, 9, 10</entry></row>
<row><entry>11</entry> <entry>9</entry> <entry>1</entry> <entry>1, 2, 
5, 7, 9, 10, 
11</entry></row>
<row><entry>...</entry> <entry>9</entry> <entry>1</entry> <entry>1, 
2, 5, 7, 9, 10, 11, 
...</entry></row>
</tbody>
</tgroup>
</table>
			
	

Un esquema elegante puede reducir la cantidad de trabajo necesaria, pero esto 
no significa que existan menos cosas a seguir. Usted decide si vale o no la pena.

	 \texttt{\textbf{dump}}
tiene soporte para copias multiniveles, pero para \texttt{\textbf{tar}} y \texttt{\textbf{cpio}}
debe ser implementado con scripts del shell.



\section{

Que copiar}

	

Debe respaldar la mayor cantidad de información posible. La mayor excepción es el
software, ya que puede ser fácilmente reinstalado.
\footnote{You get to decide what's easy.
		Some people consider installing from dozens of floppies
		easy.}
Tenga en cuenta que los programas muchas veces están configurados manualmente
para sus sistemas,
por lo que es importante respaldar a los archivos de configuración, y evitar así
la tarea de reconfigurarlos nuevamente.
Otra gran excepción es el sistemas de archivos \texttt{/proc}, debido a 
que solamente contiene información que el kernel genera automáticamente. Por lo
que nunca es buena idea copiarlo, especialmente
el archivo \texttt{/proc/kcore} es innecesario, ya que solamente es una imagen 
de la memoria física, y nunca suele ser muy útil dentro de un backup (solo ocupa 
bastante espacio).

	

Hay algunas otras partes de un sistema Linux que deben ser analizadas antes de 
ser incluidas en sus copias de respaldo. Por ej., los archivos de log, las colas de los
archivos de impresión, las news y muchas otras cosas que se encuentran bajo \texttt{/var}. 
Debe decidir que considera importante y que no.

	

Lo obvio al momento de respaldar son los archivos de los usuarios (\texttt{/home}) y 
los archivos de configuración del sistema (\texttt{/etc}, y posiblemente otros archivos
y directorios que se encuentren dispersos en el sistema de archivos).



\section{

Copias de seguridad comprimidas}

	

Las copias de seguridad ocupan una gran cantidad de espacio, por lo que
puede ser costoso al momento de invertir en el medio a utilizar.
Para reducir el espacio necesario la copias de seguridad pueden ser comprimidas. Existen varias 
formas de hacerlo, pero una de la mas práctica es utilizar directamente programas con soporte para compresión.
Por ejemplo GNU \texttt{\textbf{tar}}, cuya opción <option>--gzip</option> (<option>-z</option>) 
hace que la información resguardada sea procesada por el programa de compresión 
\texttt{\textbf{gzip}} antes de ser copiada al medio para backups.
	
	

Desafortunadamente las copias de respaldo comprimidas pueden causar 
problemas. Debido a la naturaleza de como la compresión funciona, si un simple 
bit es incorrecto, todo el resto de los datos comprimidos son 
inutilizables. Algunos programas para backups pueden corregir algunos de 
estos errores, pero ningún método de los implementados 
pueden manejar un gran número de errores. Esto significa que si la copia de 
seguridad es comprimida en la manera en que GNU \texttt{\textbf{tar}} lo hace (la salida
de la copia comprimida es una única unidad), el backup completo puede 
ser inútil si tan solo existe un simple error.
Las copias deben ser 
confiables, y este método de compresión no es una buena idea.
	
	

Una manera alternativa es comprimir cada uno de los archivos separadamente. 
El problema mencionado anteriormente aún persiste, pero si un archivo en el backup se 
encuentra dañado, al menos los demás no sufren los efectos colaterales.
El archivo perdido podría de cualquier forma tener algún otro tipo de error (en su versión original), por 
lo que esta situación no parece ser menos favorable para decidir no utilizar ningún tipo de compresión.
El programa \texttt{\textbf{afio}} (una variante del programa \texttt{\textbf{cpio}}) es capaz de trabajar de esta manera.
	
	

La compresión lleva tiempo, por lo que el programa para backups puede 
no ser capaz de escribir lo suficientemente rápido a una unidad de cinta..
	
		\footnote{If a tape drive doesn't data fast enough,
		it has to stop; this makes backups even slower, and can
		be bad for the tape and the drive.}
		

Puede evitar este problema si la
salida es mantenida en un buffer (implementado internamente si el programa es 
un poco inteligente, o utilizando algún otro programa), pero aún puede no llegar a 
trabajar lo suficientemente bien. Bajo este panorama es importante que controle
el estado de la finalización de las tareas de respaldo, y note que este último problema
solo puede suceder en computadoras lentas (o en computadoras muy fuertemente utilizadas,
en donde el programa de compresión no tenga muchas chances de utilizar la CPU a tiempo).






