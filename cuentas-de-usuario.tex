

\usepackage{fancyhdr}
        \pagestyle{fancy}
\usepackage{lastpage}
        \lhead{}
        \chead{}
        \rhead{}
        \lfoot{\footnotesize Administrando Cuentas de Usuario Linux - Departamento de Ingeniería de Computadoras}
        \cfoot{}
        \rfoot{\footnotesize page \thepage\ of \pageref{LastPage}}      % "Page 1 of 2"
        \renewcommand{\headrulewidth}{0.0pt}
        \renewcommand{\footrulewidth}{0.4pt}

\def\ti#1#2{\texttt{#1} & #2 \\ }



\begin{document}

\setlength{\parindent}{0pt}


% Titulo
\makeatletter
{\color{bl} \centering \huge \sc \textbf{
Administrando Cuentas de Usuario \\
% \large \vspace*{-8pt} \color{black} Administrando Cuentas de Usuario
\vspace*{8pt} }\par}
\makeatother


% Autor
\makeatletter
{\centering \small 
        Departamento de Ingeniería de Computadoras \\
        Facultad de Informática - Universidad Nacional del Comahue \\
        \vspace{20pt} }
\makeatother





\textit{Similitudes entre administradores de sistema y
narcotraficantes: ambos miden cosas en Kilos y tienen usuarios} (Viejo y
cansador chiste de computación)

Es necesario familiarizarse con varios aspectos relativos a las
cuentas de los usuarios, por lo que en este capítulo se detallan 
-al menos- las tareas de creación, modificación y borrado.
Estos temas se encuentran explicados
mencionando los programas y archivos generales a todos los sistemas Linux.
Pero, debido a la diversidad de distribuciones existentes,
diferentes sistemas proveen algunas otras herramientas extras para la gestión de cuentas de usuario (consulte la documentación
específica a su distribución Linux para obtener mayor información con 
respecto a este tema).

		
\section{ ¿Qué es una cuenta?}

Concretamente es un nombre de usuario -mas una contraseña, salvo
excepciones- y todos los archivos (de configuración y aquellos personales) que
impliquen el ingreso y permanencia en el sistema  de un usuario .
Cuando una computadora la usa mucha gente es usualmente necesario hacer
diferencias en estos usuarios. Por ejemplo, para que sus archivos privados
permanezcan privados. Esto es importante aun si el sistema es usado por una
sola persona a la vez, como sucede con la mayoría de las computadoras PC.
		\footnote{
 Puede ser un poco embarazoso si mi hermana pudiera leer mis cartas de amor
		.} Así, a cada
		usuario se le da un nombre de usuario único, y ese nombre es
		usado para ingresar al sistema. Un usuario es mas
		que sólo un nombre, como sea?. Una \textit{cuenta}
		es todos los archivos, recursos, e información que pertenece a
		un usuario. El término insinúa como en bancos y en sistemas
		comerciales, cada cuenta usualmente tiene algo de dinero
		asignado, y ese dinero se gasta a diferentes velocidades
		dependiendo de cuantos usuario exijan el sistema. Por ejemplo,
		el espacio de disco puede tener un precio por mega por día, y
		tiempo de procesamiento puede tener un precio por
		segundo.






\section{ Crear una cuenta de usuario}

El núcleo de Linux en sí trata a los usuario como meros números. Cada
usuario es identificado por un único numero entero, el  \textit{uid
(identificación de usuario)} esto debido a que un numero es mas fácil
y rápido de procesarlo que un nombre para un sistema. Una base de datos o tabla
asociada a dichos UIds y GIds, por fuera del núcleo asigna un nombre textual, un
único \textit{nombre de usuario} para cada id. La base de datos
que también contiene información adicional. 

Para crear un usuario, necesita agregar información sobre el usuario a la
base de datos (ver arriba) y crear un directorio "inicio" (directorio principal
del usuario) para él. También puede ser necesario educar al usuario, y
configurar un ambiente conveniente para él.

La mayoría de las distribuciones de Linux cuentan con programas para crear
cuentas. Existen varios programas disponibles en los repositorios,
o en Internet
	\footnote{En Internet puede comenzar buscando en:
	http://sourceforge.com y http://freshmeat.com}. Tal vez,
los comandos mas utilizados sean \texttt{\textbf{adduser}} y
\texttt{\textbf{useradd}}. También existen herramientas gráficas, por ejemplo,
en KDE se encuentra \texttt{\textbf{Kuser}}.
Cualquiera sea 
el programa, resulta muy poco el trabajo manual por hacer. Aun
cuando los detalles son muchos e intrincados, estos programas hace parecer todo
trivial. En la sección \textit{Crear un usuario a mano} se describe
todos los detalles relacionados.





\subsection{ \texttt{/etc/passwd} y otros archivos información \texttt{/etc/shadow}}

La base de datos básica de usuarios en un sistema Unix es un archivo de
texto  \texttt{/etc/passwd} (llamado el \textit{archivo de
contraseñas}), que lista todos los nombres de usuarios validos y su
información asociada. El archivo tiene una línea por usuario, y es dividido en
siete colon-delimited campos.


	\begin{itemize}
	
	\item{nombre de usuario}
	\item{contraseña, de modo encriptado}
	\item{Identificación (Id) de numero de usuario} 
	\item{Identificación (Id) de numero de grupo} 
	\item{Nombre completo u otra información descriptiva de la cuenta}
	\item{Directorio Inicio (directorio principal del usuario)}
	\item{Interprete de comandos (programa a ejecutar al ingresar al sistema)}

	\end{itemize}

El formato esta explicado con mas detalles en la pagina de manual del comando
\texttt{passwd}.

Cualquier usuario del sistema puede leer el archivo de contraseñas, para
por ejemplo conocer el nombre de otro usuario del mismo. Esto significa que la
contraseña (el segundo campo) esta también disponible para todos. El archivo de
contraseñas encripta las contraseñas, así que en teoría no hay problema, pero
dicho encriptado puede ser quebrado, sobre todo si dicha contraseña es
"débil"

	\footnote{ Estadísticamente, según el estudio de
los métodos para romper claves encriptadas, se ha establecido que aumenta
significativamente la seguridad una suma de características: tener mas de 6
caracteres, combinar letras mayúsculas y minúsculas, a la vez que intercalar
también números.}.  Por lo tanto no es buena idea tener las contraseñas en el archivo
de contraseñas.

Muchos sistemas GNU/Linux tienen contraseñas
\textit{"sombra"}.Esto es una  alternativa en la manera de
almacenar las contraseñas: las claves encriptadas se guardan en un archivo
separado \texttt{/etc/shadow} que solo puede ser leído por el
administrador del sistema. Así el archivo \texttt{/etc/passwd} solo
contiene un marcador especial en ese segundo campo. Cualquier programa que
necesite verificar un usuario o uid, pueden también acceder al archivo
shadow/sombra.  Significa también que programas normales que solo usan otros
campos del archivo de contraseñas, no pueden acceder a las contraseñas.
Paralelamente también existe \texttt{/etc/gshadow} para cierta
información según grupos	\footnote{
 Si, esto significa que en el archivo de contraseña contiene toda la
información sobre un usuario \textit{\bf excepto} su contraseña. Una maravilla del
desarrollo.}

	
	


\subsection{  Elegir números de identificación de usuario y grupo}

En la mayoría de los sistemas, no importa cuales son los números de
usuario y grupo, pero si usa un sistema de archivos de red 
	\footnote{NFS: Network file System}, necesitará que
sean los mismo números de identificación de usuario (uid) y grupo (gid) en todos
los sistemas. Esto es porque el sistema de archivos de red también identifica al
usuarios (nombre de usuario) con su ++respectivo++ numero de identificación de
usuario (uid).  Si esta usando un sistema de archivos de red NFS
\footnote{NFS: Network file System}, tiene que inventar un mecanismo para sincronizar la información de cada
cuenta. Una alternativa es el sistema NIS
	\footnote{NIS:
	Network Information Service}(ver Guía de Administración de
Redes con Linux, Capítulo 13) Como sea, trate de evitar re-usar
números de identificación de usuario (UIds) y nombres de usuario exactamente
iguales entre si entre sistemas, porque el nuevo dueño de ese numero de
identificación de usuario o nombre de usuario puede tener (o tendrá seguro?)
acceso a los archivos o correos-e del anterior dueño.








\subsection{Ambiente inicial: \texttt{/etc/skel}}
 	\footnote{ Apocope de la palabra inglesa
 	skeleton, que en castellano significa esqueleto, asiendo referencia al función
 de estructura.}

Cuando el directorio Inicio para un nuevo usuario es creado es
inicializado por medio del directorio \texttt{/etc/skel}. El
administrado del sistema puede crear archivos dentro de
\texttt{/etc/skel} que proveerán un amable entorno predeterminado
para los usuarios. Por ejemplo, el puede crear un
\texttt{/etc/skel/.profile} que configura  las variable de entorno
de algún editor mas amigable para los usuarios nuevos.


Como sea, usualmente lo mejor es conservar dicho directorio lo mas pequeño
que sea posible, ya que en el futuro será imposible actualizar los archivos de
los usuarios. Por ejemplo, si cambia el nombre de un editor a uno mas amigable,
todos los usuarios tendrán que editar su archivo  \texttt{.profile}.
El administrador del sistema podría tratar de hacer esto automáticamente con un
script \footnote{Lenguaje de programación cuyo código no necesita ser
compilado para ser ejecutado, por lo general es interpretado por el shell.
Ver también "Expresiones regulares".}, pero
	casi con seguridad resultará que se corrompa el archivo de
alguno.  Siempre que sea posible, es mejor poner lo que sea configuración global
dentro de archivos globales, como es /etc/profile. De esta manera es posible
actualizarlo sin corromper la configuración de ningún usuario.  






\subsection{Crear un usuario a mano}

Para crear una nueva cuenta a mano, sigue estos pasos:

\begin{itemize}
	
	\item{ Editar \texttt{/etc/passwd} con
	\texttt{\textbf{vipw}} y agregar una nueva linea por cada nueva cuenta.
	Teniendo cuidado con la sintaxis.  \textit{\bf No lo edite directamente
	con un editor!}. Utilice el comando \texttt{\textbf{vipw}}, el cual bloquea el
	archivo, así otros comandos no tratarán de actualizarlo al mismo tiempo.
	Debería hacer que el campo de la contraseña sea `\texttt{*}',
	de esta forma es imposible ingresar al sistema.}

	\item{ Similarmente, edite \texttt{/etc/group} con
	\texttt{\textbf{vigr}}, si necesita crear también un
	grupo.} \item{ Cree el directorio Inicio del
	usuario con el comando \texttt{\textbf{mkdir}}.}
	\item{ Copie los archivos de \texttt{/etc/skel} al
	nuevo directorio creado 
	\footnote{cp /etc/skel/* /ruta
(donde ruta será por convención /home/"nombre de usuario"}}
	\item{ Corrija
	la pertenencia del dueño y permisos con los comandos
	\texttt{\textbf{chown}} y \texttt{\textbf{chmod}} (Ver paginas de
	manual de los respectivos comandos). La opción \texttt{-R} es
	muy útil. Los permisos correctos varían un poco de un sitio a otro, pero
	generalmente los siguientes comandos harán lo correcto:


\begin{verbatim}
cd /home/nuevo-nombre-de-usuario
chown -R nombre-de-usuario.group .  
chmod -R go=u,go-w .  
chmod go= .
\end{verbatim}


	}
	
	\item{ Asigne una contraseña con el comando
	\texttt{\textbf{passwd}}}

\end{itemize} 
	
	Después de asignar la contraseña del usuario en el ultimo paso, la
	cuenta funcionara. No debería configurar esto hasta que todo lo demás
	este hecho, de otra manera el usuario puede inadvertidamente ingresar al
	sistema mientras copias los archivos de configuración de su entorno de
	trabajo.

A veces es necesario crear cuentas "falsas"
		
		\footnote{¿Usuarios Surrealistas?} que no son
		usadas por personas. Por ejemplo, para configurar un servidor
		FTP
		\footnote{FTP: File
		transfer Protocol.} anónimo (así cualquiera podrá acceder a los archivos por
		él, sin tener que conseguir una cuenta de usuario en el sistema
		primero) podría crear una cuenta llamada "ftp". En esos casos,
		usualmente no es necesario asignar una contraseña (el ultimo
		paso de arriba).  Verdaderamente, es mejor no hacerlo, para que
		nadie puede usar la cuenta, a menos que primero sea root/cuenta
		administrador, y así convertirse en cualquier usuario.







 \section{ Cambiar las propiedades del usuario}

\begin{itemize}
	\item Hay algunos comandos para cambiar varias propiedades de cualquier cuenta: \\ \\
\begin{tabular}{ l l }
	\texttt{chfn} & cambia el campo del nombre completo \\
	\texttt{chsh} & cambia el campo del interprete de comandos \\
	\texttt{passwd} & cambia la contraseña \\
\end{tabular}
\end{itemize}


Normalmente los usuarios solo pueden cambiar las propiedades de sus propias
cuentas. A veces es necesario deshabilitar estas posibilidades (por medio del
comando \texttt{\textbf{chmod}})  para los usuarios normales, por ejemplo en un
ambiente con muchos usuarios novatos.

Otras tareas pueden ser necesarias hacerlas manualmente. Por ejemplo,
cambiar el nombre de usuario, editando el archivo
\texttt{/etc/passwd} directamente (recuerda hacerlo con el
\texttt{\textbf{vigr}}). También para agregar o quitar a uno o varios usuarios
de uno o mas grupos, editando \texttt{/etc/group} (con
\texttt{\textbf{vigr}}). Este tipo de tareas tienden a ser mas raras, de todas
maneras, siempre hay que ir con cuidado: si cambia un nombre de usuario, dicho
usuario dejara de acceder a su cuenta de correo a menos que también le genere un
alias a su dirección de correo.
	
		\footnote{Los usuarios pueden cambiar su
		nombre por haberse casado, por ejemplo, y quieran tener su cuenta de usuario
		actualizada para reflejar su nuevo nombre.}
		









\section{Borrando usuarios}

Para borrar un usuario, primero borre los archivos que le pertenezcan,
casilla de correo, alias de correo, trabajos de impresión, trabajos pendientes a
través de los demonios \texttt{\textbf{cron}} y \texttt{\textbf{at}}, y
cualquier otra referencia al usuario.  Entonces quite las correspondientes
lineas relevantes de los archivos \texttt{/etc/passwd} y
\texttt{/etc/group}	 (recuerde borrar al usuario de todos los grupos
a los cuales pertenecía).  Puede ser buena idea deshabilitar la cuenta antes de
empezar a borrar cosas para prevenir que el usuario use la cuenta mientras esta
siendo eliminado.

Recuerde que los usuarios pueden tener archivos fuera de su directorio
Inicio.  Para encontrarlos use el comando:

 find / -user username 

Como sea, note que el comando find puede tomar \textit{\bf mucho tiempo} si
tiene discos muy grandes o si monta un disco de red.


Algunas distribuciones tiene comandos especiales para realizar esta tarea. Por 
ejemplo
\texttt{\textbf{deluser}} o \texttt{\textbf{userdel}}. Igualmente, es fácil
hacerlo a manualmente, y de todas maneras puede que el comando no lo haga todo.






\section{ Deshabilitar un usuario temporalmente}

A veces es necesario deshabilitar una cuenta temporalmente, sin borrarla.
Por ejemplo, un usuario pudo dejar de pagar sus cuentas, o el administrador de
sistema puede sospechar que un cracker
	\footnote{En este caso traducible a invitado inesperado y
	malicioso} tiene la contraseña de esa
cuenta.

La mejor manera de deshabilitar una cuenta es cambiar su intérprete de
comandos por otro programa que solo envía mensajes a la pantalla. De esta manera,
cualquiera sea la forma que intente entrar al sistema con esa cuenta fracasará,
y sabrá porqué. El mensaje puede decir que el usuario se contacte con el
administrador de sistema para que cualquier problema sea tratado con/por
él. Es posible también cambiar el nombre de usuario o contraseña
del mismo por otro, en tal caso el usuario no sabrá que pasa. Usuarios confusos
significa mas trabajo.  \footnote{Pero ellos pueden llegar a ser
\textit{muy} divertidos, si eres un BOFH.}

	
	

 Una manera simple de para crear un programa especial es escribir una
"cola de script": 

 
\texttt{\#!/usr/bin/tail +2 }
Esta cuenta ha sido cerrada por razones de seguridad.
Por favor llame al 555-1234 y espere que lleguen los hombres de negro.



Los primeros dos caracteres ('\texttt{\#!}') le dicen al núcleo que el
resto de la línea es un comando que necesita ejecutarse por medio de un
interprete. El comando \texttt{\textbf{tail}} en este caso manda una salida en
pantalla de todo excepto de la primera línea de la salida estándar.

El usuario billg
	\footnote{ billg: referencia a
	Bill Gate, uno de los cofundadores de la empresa Microsoft.} es sospechado
	de infringir la seguridad, el
administrador del sistema puede hacer algo como esto: 


\begin{verbatim}
# chsh -s /usr/local/lib/no-login/security billg
# su - tester

    Esta cuenta ha sido cerrada por razones de seguridad.
    Por favor llame a 555-1234 y espere hasta que lleguen los Hombre de Negro 

# 
\end{verbatim}


El propósito del comando \textbf{su} es verificar que el cambio
funciona, por supuesto.  

"Tail script" debe mantenerse en un directorio separado, así sus nombres
no interfieren con los comandos de los usuarios normales.






\end{document}
