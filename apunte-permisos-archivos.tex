%%% LaTeX Template: Article/Thesis/etc. with colored headings and special fonts
%%%
%%% Source: http://www.howtotex.com/

\documentclass[12pt]{article}


\usepackage{apuntes-estilo}
\usepackage{fancyhdr,lastpage}



\def\maketitle{

% Titulo 
 \makeatletter
 {\color{bl} \centering \huge \sc \textbf{
Visión General del Árbol de Directorios \\
\large \vspace*{-8pt} \color{black} Una introducción al Sistema de Archivos GNU/Linux
 \vspace*{8pt} }\par}
 \makeatother


% Autor
 \makeatletter
 {\centering \small 
 	Departamento de Ingeniería de Computadoras \\
 	Facultad de Informática - Universidad Nacional del Comahue \\
 	\vspace{20pt} }
 \makeatother

}

% Custom headers and footers
\fancyhf{} % clear all header and footer fields
\fancypagestyle{plain}{\fancyhf{}}
  	\pagestyle{fancy}
 	\lhead{\footnotesize Visión General del Árbol de Directorios - Departamento de Ingeniería de Computadoras}
 	\rhead{\footnotesize \thepage\ }	% "Page 1 of 2"

\def\ti#1#2{\texttt{#1} & #2 \\ }



\begin{document}

\thispagestyle{empty}
\maketitle
\setlength{\parindent}{0pt}


\section{ Introducción}
Este documento describe los conceptos básicos sobre propiedades de archivos, 
incluyendo tipos y permisos de archivos, propietarios y grupos. 

\section*{Tipos de archivos}

Un archivo es una secuencia de bytes (un byte es un pequeño trozo de 
información, normalmente compuesto por ocho bits. Para nuestros propósitos
durante este texto, un byte es equivalente a un caracter). El sistema no impone 
estructura alguna a los archivos, ni asigna significado a su contenido; el 
significado de los bytes depende únicamente de los programas que 
interpretan el archivo. 

Vimos en apuntes anteriores que los archivos se organizan dentro de sistemas de 
archivos (ext3, ext4,NTFS, etc). Estos sistemas de archivos son el modo de 
ordenar los archivos dentro de los distintos medios de almacenamiento (discos
rígidos, de estado sólido, memorias flash, etc). 

En general, la mayoría de los sistemas de archivos de sistemas de tipo UNIX
definen seis tipo de archivos:

\begin{itemize}
\item Archivos regulares (Ej. un archivo de texto, un pdf son archivos regulares)
\item Directorios
\item Links simbólicos
\item Dispositivos (de bloque y caracter) 
\item UNIX Sockets  
\item Pipes 
\end{itemize}

Limitaremos nuestro estudio a los tres primeros tipos de archivos. 


\subsection*{Archivos regulares}
No son otra cosa que una bolsa de bytes: UNIX no impone ninguna estructura a
su contenid. Los archivos de texto, de datos, programas ejecutables, bibliotecas
compartidas, son todos archivos regulares.

La creación de archivos regulares sucede a través de distintos programas de 
aplicación, por ejemplo cuando utilizamos un sofware de ofimática, cuando 
descargamos archivos del mail, a través de un editor de texto, etc. El formato 
inerno de archivos regulares dependerá entonces de la aplicación que lo creó. 

La eliminación de archivos regulares sucede a través del comando \texttt{rm}. 

\begin{verbatim}
-rw-r--r--  1 lechnerm lechnerm     46887 mar 28 16:19 tivoli.png
-rw-r--r--  1 lechnerm lechnerm   6682574 abr  1 18:38 tmp.79167916.bmp
-rw-r--r--  1 lechnerm lechnerm    126463 sep 21  2013 todosej2.txt.pdf
-rw-r--r--  1 lechnerm lechnerm     95766 dic 17 10:55 vclab .odt
\end{verbatim}

\subsection*{Directorios}
Los directorios son archivos que contienen referencias a otros archivos. 
Podemos crear directorios a través del comando \texttt{mkdir}, y eliminarlos
a tevés del comando \texttt{rmdir} si está vacío, o \texttt{rm -r} si tiene 
contenido. 

La estructura de un archivo de tipo directorio dependerá del sistema de 
archivos que usemos en particular. En general las entradas ``.'' y ``..''
se refieren al directorio actual o su padre respectivamente.  

Los nombres de archivos contenidos dentro de un directorio, se encuentran 
almacenados en el directorio y no en el archivo. Esto permite entre otras
cosas que un archivo exista en varios directorios simultáneamente. 

\section{ El sistema de archivos raíz}

 El sistema de archivos raíz debería ser pequeño, ya que residen archivos
muy críticos. Si el sistema de archivos es pequeño y rara vez es modificado,
tiene menos posibilidades de sufrir daños. Un sistema de archivos raíz dañado,
generalmente significa que el sistema no podrá arrancar a no ser que se tomen
medidas especiales (P.Ej., tal vez pueda arrancar desde un CD-ROM o pendrive de
emergencia), por lo que no se desea correr el riesgo.  

 El directorio\footnote{Nótese que se está hablando del directorio y no del 
sistema de archivos en su totalidad como en el párrafo anterior.} raíz no 
contiene generalmente archivos, exceptuando quizás
la imagen del núcleo estándar, normalmente llamada
\texttt{/vmlinuz}. Todos los demás archivos se encuentran en
subdirectorios bajo el sistema de archivos raíz:

\begin{itemize} 

	\item \textbf{\texttt{/bin}}
	 Comandos necesarios durante el inicio del sistema que
	pueden ser utilizados por usuarios normales (probablemente después de
	que el sistema haya arrancado).  

	\item \textbf{\texttt{/sbin}}
	 Igual que \texttt{/bin}, pero aquí los
	comandos no están destinados a los usuarios normales, aunque
	pueden utilizarse en caso de que sea necesario y el sistema lo permita.
	\texttt{/sbin} no se encuentra en las rutas de acceso (PATH) de manera 
	predeterminada de los usuarios normales. Por el contrario, sí se 
	encuentra definido en la ruta predeterminada para el usuario root.
	
	\item \textbf{\texttt{/etc}}
	 Archivos de configuración específicos de la máquina (P.Ej., la base
	de datos de usuarios locales \texttt{/etc/passwd}.
	
	\item \textbf{\texttt{/root}}
	 El directorio local o \texttt{home} para el usuario root.  Normalmente
	los demás usuarios del sistema no pueden acceder a él.
	
	\item \textbf{\texttt{/lib}}
	 Bibliotecas compartidas necesarias para los programas que
	se encuentran en el sistema de archivos raíz.
	
	\item \textbf{\texttt{/lib/modules}}
	 Módulos cargables del núcleo, especialmente
	aquellos que se necesitan para arrancar el sistema tras recuperarse
	de algún incidente (P.Ej., controladores de red y sistemas de
	archivos).

	\item \textbf{\texttt{/dev}}
	 Archivos de dispositivos (P.Ej., /dev/sda puede representar al primer
	disco local del sistema, estos nombres pueden variar de una distribución
	a otra).

	\item \textbf{\texttt{/tmp}}
	 Archivos temporales. Los programas que se ejecuten
	después de que el sistema se haya iniciado deben utilizar
	\texttt{/var/tmp}, no \texttt{/tmp},
	debido a que \texttt{/var/tmp} probablemente resida en una
	partición o disco con más espacio. Frecuentemente /tmp es un enlace
	simbólico para /var/tmp.  

	
	\item \textbf{\texttt{/boot}}
	 Archivos utilizados por el gestor de arranque, por
	ejemplo, GRUB o LILO. Las imágenes del núcleo o kernel se guardan con
	frecuencia en este directorio, en vez de en el directorio raíz. Si
	existen	muchas imágenes del núcleo, el directorio puede
	llegar a crecer mucho, por lo que es mejor mantener este
	directorio en un sistema de archivos separado. 
	
	\item \textbf{\texttt{/mnt}}
	 Punto de montaje temporal para los sistemas de archivos
	montados manualmente por el administrador del sistema. Se supone que los programas
	no deben montar en \texttt{/mnt} automáticamente. Es posible
	que \texttt{/mnt} se encuentre dividido en subdirectorios
	(P.Ej.,\texttt{/mnt/usb0} puede ser
	el punto de montaje para los pendrive con sistema de archivos
	FAT, y \texttt{/mnt/cdrom} puede llegar a serlo para CD-ROM con sistemas
	de archivo ISO9669).  
	
	\item \textbf{\texttt{/proc}
	\texttt{/usr} \texttt{/var}
	\texttt{/home} } Son puntos de
	montaje para otros sistemas de archivos.
 \end{itemize}  




\section{ El directorio /etc}

 El directorio \texttt{/etc} contiene gran cantidad de
archivos de configuración del sistema. Algunos de ellos se describen aquí. 
Para otros archivos, se
debe determinar a qué programa pertenecen y leer la página de manual
correspondiente.  

\begin{itemize} 

\item \textbf{\texttt{/etc/rc} o
\texttt{/etc/rc.d} o \texttt{/etc/rc?.d}}
Scripts\footnote{Programas escritos en el lenguaje de programación definido por 
un shell, por ejemplo, Bash} o 
directorios de scripts que se ejecutan durante el
arranque del sistema o al cambiar el nivel de ejecución. Se puede
encontrar información adicional en la documentación dedicada a Init (\texttt{man init} y 
\texttt{man inittab}).

\item \textbf{\texttt{/etc/passwd}}
 Es la base de datos de los usuarios locales, que incluye campos como el
nombre de usuario, nombre real, identificador de usuario (UID), directorio home, y otra
información acerca de cada usuario.  El formato de este archivo se encuentra
documentado en la página del manual \texttt{\textbf{passwd}} dentro
de la sección cinco\footnote{\texttt{man passwd} hará referencia a la sección uno
del manual, mostrando el manual para el comando utilizado para cambiar una contraseña} 
(\texttt{man 5 passwd}). 

\item  \textbf{\texttt{/etc/shadow}}
 En este archivo se encuentra la información de las contraseñas de usuarios locales, junto 
a información relativa a dichas contraseñas, como por ejemplo fecha del último cambio
de contraseña (\texttt{man shadow}). No debe poder ser leído por nadie a excepción 
del usuario root. De esta manera se dificulta el proceso de descifrado de las contraseñas 
de los usuarios (si fuera accesible por cualquier usuario un ataque por fuerza bruta podría
revelar una o más contraseñas de usuarios).

\item \textbf{\texttt{/etc/group}}
 Este archivo es similar a \texttt{/etc/passwd} , pero describe grupos locales
de usuarios. Se puede encontrar información adicional en la página de manual 
\texttt{\textbf{group}} (\texttt{man shadow}).  

\item \textbf{\texttt{/etc/fstab}}
   Lista los sistemas de archivos montados automáticamente
durante el arranque del sistema por el comando \texttt{\textbf{mount
-a}} (en \texttt{/etc/rc} o archivo de inicio
equivalente). En Linux, este archivo también
contiene información acerca de áreas de swap utilizadas
automáticamente por \texttt{\textbf{swapon -a}} (\texttt{man fstab}).  

\item \textbf{\texttt{/etc/mtab}}
Contiene un listado de los sistemas de archivos
actualmente montados. Se establece Inicialmente por los scripts del
arranque del sistema, y se actualiza automáticamente por el comando
\texttt{\textbf{mount}}. Se utiliza cuando se necesita un listado de
los sistemas de archivos que estén actualmente montados (por ejemplo por
el comando \texttt{df}).  

\item \textbf{\texttt{/etc/inittab}}
Archivo de configuración para el proceso init \texttt{man inittab}).
	
\item \textbf{\texttt{/etc/issue}}
 Archivo de texto que utiliza \texttt{\textbf{getty}} como
salida antes de que el sistema pida el nombre de usuario. Usualmente
contiene una descripción corta o mensaje de bienvenida al sistema.  El
contenido es establecido por el administrador del sistema. También suele
utilizarce en la configuración de otros programas que permiten inicios 
de sessión como el servidor Open-SSH (vea \texttt{man sshd\_config}, 
opción ``Banner''). 
	
\item \textbf{\texttt{/etc/motd}}
 Contiene el mensaje del día, que se emite
automáticamente tras iniciar una sesión con éxito. El
contenido es definido por el administrador del
sistema. Con frecuencia se utiliza para dar información a todos
los usuarios, como por ejemplo, mensajes de advertencias acerca de la
hora en que está planeada una parada técnica del servidor.

\item \textbf{\texttt{/etc/magic}}
El archivo de configuración para el programa
\texttt{\textbf{file}}. Contiene las descripciones de varios formatos
de archivos que utiliza file para determinar el tipo de archivo. Se
puede encontrar información adicional en las páginas de manual
para \texttt{magic} y \texttt{\textbf{file}}.
	

	
\item \textbf{\texttt{/etc/profile},
\texttt{/etc/csh.login},
\texttt{/etc/csh.cshrc}}  Archivos
que se ejecutan en el momento de iniciar los intérpretes de comandos C o
Bourne. Permite al administrador del sistema establecer parámetros
globales de manera predeterminada para todos los usuarios. Se puede encontrar
información adicional en las páginas de manual para los respectivos
intérpretes de comandos.  


\item \textbf{\texttt{/etc/securetty}}
 Identifica las terminales seguras, esto es, las
terminales por las cuales el usuario root tiene permitido iniciar una
sesión. Típicamente sólo las consolas virtuales se encuentran listadas
en este archivo, con lo que se hace imposible (o al menos mas difícil)
obtener privilegios de superusuario accediendo a través de un módem o la
red. No se debe permitir iniciar una sesión como usuario root desde la
red. Es preferible iniciar una sesión con un usuario sin privilegios y
utilizar después \texttt{\textbf{su}} o \texttt{\textbf{sudo}} para
obtener privilegios de superusuario.  

\item \textbf{\texttt{/etc/shells}}
 Listado de intérpretes de comandos admitidos.  El
comando \texttt{\textbf{chsh}} permite a los usuarios cambiar su
intérprete de comandos predeterminado a otro que se encuentre listado en
este archivo. Relacionado a este archivo, \texttt{\textbf{ftpd}}, el 
proceso servidor que brinda servicios de transferencias de archivos (FTP), 
comprueba que los intérpretes de comandos de los usuarios estén listados
en \texttt{/etc/shells}, caso contrario el servicio es denegado.  

\end{itemize} 
    


\section{ El directorio /dev}

El directorio \texttt{/dev} contiene los archivos de
dispositivos especiales para muchos dispositivos de hardware. Los archivos de
dispositivos se nombran utilizando convenciones especiales.  Algunos de estos
archivos se crean durante la instalación del sistema, otros son creados 
dinámicamente por el sistema udev (\texttt{man udev}) y, en casos extraordinarios,
pueden ser creados manualmente por el administrador de sistemas.

El sistema udev (y su correspondiente demonio \texttt{udevd}) crean archivos de
dispositivos basados en eventos que recibe del kernel, a medida que se agregan 
o remueven dispositivos del sistema. 

\fcolorbox{black}{grey}{
\parbox[t]{1.0\linewidth}{ \vspace*{0.4cm}
{\bf Algunos ejemplos de archivos de dispositivo:}
\begin{itemize}
\item \texttt{/dev/sda} primer disco SCSI del sistema. Mientras que \texttt{/dev/sda1} 
es el archivo que representa a la primer partición del disco. 
\item \texttt{/dev/tty?} (de 0 a 63) son los archivos que representan a las terminales 
del sistema. 
\end{itemize}
\vspace*{0.4cm} } }

\section{ El sistema de archivos /usr}
El sistema de archivos \texttt{/usr} es con frecuencia grande,
debido a que todos los programas están instalados allí. Normalmente, todos los
archivos en \texttt{/usr} provienen de la distribución Linux que
hayamos instalado; los programas instalados localmente y algunas otras cosas
se encuentran bajo \texttt{/usr/local}. De esta manera es posible
actualizar el sistema desde una nueva versión de la distribución, o incluso
desde una distribución completamente nueva, sin tener que instalar todos los
programas nuevamente. Algunos de los directorios de \texttt{/usr}
están explicados aquí debajo (algunos de los menos importantes se han omitido,
se puede encontrar información adicional en el Estándar del Sistema de
Ficheros).  


\begin{itemize} 
	
	\item \textbf{\texttt{/usr/X11R6}}
	 Se encuentran aquí todos los archivos del Sistema
	X-Windows. Para simplificar el desarrollo y la instalación de X,  sus
	archivos no fueron integrados dentro del resto del sistema. Existe un
	árbol de directorios bajo /usr/X11R6 similar al que está bajo /usr.
	


	\item \textbf{\texttt{/usr/bin}}
	 En este directorio se encuentran la gran mayoría de los
	comandos para los usuarios. Algunos otros comandos pueden encontrarse en
	\texttt{/bin} o en 	\texttt{/usr/local/bin}.
	

	\item \textbf{\texttt{/usr/sbin}}
	 Comandos para la administración del sistema que no son
	necesarios en el sistema de archivos raíz, como por ejemplo la mayoría
	de los programas que proveen servicios.  

	\item \textbf{\texttt{/usr/share/man},
	\texttt{/usr/share/info},
	\texttt{/usr/share/doc}}  Páginas
	de manual, documentos de información GNU, y 		archivos de
	documentación de los programas instalados.
	

	\item \textbf{\texttt{/usr/include}}
	 Archivos cabecera para el lenguaje de programación C.
	Estos deberían estar de hecho debajo de \texttt{/usr/lib}
	por coherencia, pero tradicionalmente se ha apoyado de forma mayoritaria
	esta ubicación.  

	\item \textbf{\texttt{/usr/lib}}
	 Archivos de datos de programas y subsistemas que no
	sufren cambios, incluyendo algunos archivos de configuración globales.
	El nombre lib viene de biblioteca (library); originariamente las
	biblotecas de las subrutinas de programación se
	almacenaban en \texttt{/usr/lib}.
	

	\item \textbf{\texttt{/usr/local}}
	 Es la ubicación para el software construído e instalado localmente, y
	para algunos otros archivos.  Las distribuciones no deben colocar
	archivos bajo este directorio. Se 			reserva para ser
	utilizado únicamente por el administrador local del
	sistema. De esta manera, aquel se asegura totalmente de que ninguna
	actualización de su distribución sobreescribirá  el software que él
	mismo haya instalado localmente.


 \end{itemize} 




\section{ El sistema de archivos /var}
    
    El sistema de archivos \texttt{/var} contiene datos que
    cambian cuando el sistema se ejecuta normalmente. Es específico para cada
    sistema y por lo tanto no es compartido a través de la red con otras
    computadoras. 

	\begin{itemize}  
	\item 
	\textbf{\texttt{/var/log}} 

	  Archivos de log (en español bitácora) de diferentes
	 programas, especialmente de \texttt{\textbf{login}}
	 (\texttt{/var/log/wtmp}, el cual registra todos los inicios
	 y cierres de sesión en el sistema) y de \texttt{\textbf{syslog}}
	 (\texttt{/var/log/messages}, en donde se almacenan todos
	 los mensajes del núcleo y de los programas del sistema). Los Archivos
	 dentro del directorio \texttt{/var/log} pueden crecer
	 indefinidamente, por lo que se requiere una limpieza a intervalos
	 regulares.   

	\item 
        
	\textbf{\texttt{/var/cache/man}} 
        
	 Actúa como una caché para las páginas de manual que son
	formateadas bajo demanda. Las fuentes de las páginas de manual se
	encuentran almacenadas normalmente en \texttt{/usr/man/man?}
	(donde ? es la sección de las páginas de manual que corresponda. Se
	puede 	encontrar información adicional en la página de manual para
	\texttt{\textbf{man}}); algunas páginas de manual
	pueden 	llegar a venir con una versión pre-formateada, la cual estaría
	almacenada 	en \texttt{/usr/share/man/cat*}. Otras
	páginas de manual necesitan ser formateadas al ser visualizadas por
	primera vez; la versión formateada es almacenada entonces en
	\texttt{var/cache/man} para que la próxima vez que un
	usuario necesite ver la misma página no tenga que esperar a que se le de
	formato.  

	\item 
        
	\textbf{\texttt{/var/games}} 
        
	 Cualquier información variable perteneciente a juegos
	existente en \texttt{/usr} debería colocarse aquí. Esto es
	así por si se da el caso de que \texttt{/usr} esté montado
	como 		solo lectura 

	\item 

	\textbf{\texttt{/var/lib}} 
    
	 Contiene archivos que cambian mientras el sistema se
	ejecuta normalmente. 

	\item 
        

	\textbf{\texttt{/var/local}} 
        
	 Datos variables de los programas que se encuentran
	instalados en \texttt{/usr/local}(aquellos que fueron
	instalados localmente por el administrador del sistema). Conviene
	reseñar  		que aunque los programas se encuentren
	instalados localmente, deben 			utilizar los otros
	directorios \texttt{/var} en caso de ser
	apropiado, como por ejemplo:
	\texttt{/var/lock}.  

	\item 
	\textbf{\texttt{/var/lock}} 
        
	 Archivos de bloqueo. Muchos programas siguen una
	convención para crear un archivo de bloqueo en /var/lock que indique que
	están utilizando un dispositivo particular o un archivo de forma
	exclusiva. Así, Los demás programas se encontrarán con el archivo de
	bloqueo y no intentarán acceder al dispositivo o archivo.
	  


	\item 
        
	\textbf{\texttt{/var/mail}} 
        
	 Esta es la ubicación definida por el FHS (Estándar de la
	jerarquía del sistema de archivos) para almacenar los archivos de buzón
	de correos de los usuarios. Dependiendo de en qué 		grado su
	distribución cumpla con el FHS, estos archivos pueden 	llegar a
	estar ubicados en \texttt{/var/spool/mail}.
	  

	\item 

	\textbf{\texttt{/var/run}} 
        
	 Directorio que contiene archivos con información acerca
	del sistema, la cual es válida hasta el próximo inicio del mismo. Por
	ejemplo: \texttt{/var/run/utmp} contiene información de las
	personas que actualmente tienen sesiones iniciadas.  
	 

	\item 
        
	\textbf{\texttt{/var/spool}} 
	 Contiene directorios para las noticias, el correo, colas
	de impresión, y otros programas que necesiten trabajar con colas.
	Cada spool diferente tiene su propio directorio debajo de
	\texttt{/var/spool}, por ejemplo: el spool de noticias se
	encuentra en \texttt{/var/spool/news}. Cabe destacar que si
	alguna instalación no cumple totalmente con la última versión del
	FHS, los buzones de correo entrante de los usuarios pueden encontrarse
	en /var/spool/mail.   

	\item 
        
	\textbf{\texttt{/var/tmp}} 
	 Archivos temporales grandes o que necesitan existir por
	un tiempo más amplio de lo permitido en \texttt{/tmp}.  (De
	todas formas, el administrador del sistema puede no permitir muchos
	archivos antiguos en /var/tmp si así lo desea).  
	  \end{itemize}  


\section{ El sistema de archivos /proc}

El sistema de archivos \texttt{/proc} contiene un sistema de
archivos imaginario o virtual. Este no existe físicamente en disco, sino que el
núcleo lo crea en memoria. Se utiliza para ofrecer información relacionada con
el sistema (originalmente acerca de procesos, de aquí su nombre). Algunos de los
archivos más importantes se encuentran explicados mas abajo. El sistema de
archivos \texttt{/proc} se encuentra descrito con más detalle en la
página de manual de proc.

	\begin{itemize} 
	
	\item
	\textbf{\texttt{/proc/1}}
	 Un directorio con información acerca del proceso con identificador
	número uno, init. Cada proceso tiene un directorio debajo de /proc 
        cuyo nombre es el número de identificación del proceso (PID).  
	
	
	\item
	\textbf{\texttt{/proc/cpuinfo}}
	 Información acerca del procesador: su tipo, marca,
	modelo,	rendimiento, etc.  
	
	\item \textbf{\texttt{/proc/meminfo}}
	 Información acerca de la memoria
	física y su utilización. 
	
	\item
	\textbf{\texttt{/proc/devices}}
	 Lista de controladores de dispositivos configurados
	dentro 	del núcleo que está en ejecución.
	
	\item
	\textbf{\texttt{/proc/filesystems}}
	 Lista los sistemas de archivos que están soportados por
	el kernel.  
	
	
	\item \textbf{\texttt{/proc/loadavg}}
	 El nivel medio de carga del sistema; tres indicadores
	significativos sobre la carga de trabajo del sistema en cada
	momento. Dichos valores son utilizados por varios comandos, como por 
	ejemplo \texttt{uptime} y \texttt{top}. 
	
	
	\item
	\textbf{\texttt{/proc/modules}}
	 Indica los módulos del núcleo que han sido cargados
	hasta el momento.
	

	\item \textbf{\texttt{/proc/net}}         
	 Información acerca del estado de los protocolos de
	red.
	
	\item \textbf{\texttt{/proc/stat}}
	 Varias estadísticas acerca del sistema, tales como el
	número de fallos de página que han tenido lugar desde el arranque del
	sistema.
	
	\item \textbf{\texttt{/proc/uptime}}
	 Indica el tiempo en segundos que el sistema lleva
	funcionando.
	
	\item \textbf{\texttt{/proc/version}}
	 Indica la versión del núcleo
	
	
	 \end{itemize} 

    Conviene aclarar que aunque los archivos anteriores tienden a ser
    archivos de texto fáciles de leer, algunas veces pueden tener un formato
    que no sea fácil de interpretar. Por ello existen muchos comandos que
    solamente leen los archivos anteriores y les dan un formato distinto para
    que la información sea fácil de entender. Por ejemplo, el comando
    \texttt{\textbf{free}}, lee el archivo \texttt{/proc/meminfo}
    y convierte las cantidades dadas en bytes a kilobytes (además de agregar 
    información extra). 
\section*{Licencia}
This is a derived version of "Guía Para Administradores de Sistemas GNU/Linux" that 
can be found at \\ \texttt{http://www.ibiblio.org/pub/Linux/docs/LDP/system-admin-guide/translations/es/gasl.txt}.

Copyright (C) 1993-1998 Lars Wirzenius.

Copyright (C) 1998-2001 Joanna Oja.

Copyright (C) 2001-2003 Stephen Stafford.

Copyright (C) 2003-2004 Stephen Stafford and Alex Weeks.

Copyright (C) 2004-Present Alex Weeks.

Copyright (C) 2013-Present Rafael Ignacio Zurita

Trademarks are owned by their owners.

Permission is granted to copy, distribute and/or modify this document under
the terms of the GNU Free Documentation License, Version 1.2 or any later
version published by the Free Software Foundation; with no Invariant
Sections, no Front-Cover Texts, and no Back-Cover Texts. A copy of the
license is included in the section entitled "GNU Free Documentation License".

\subsection*{Traducción de la licencia}
Esta es una obra derivada de "Guía Para Administradores de Sistemas GNU/Linux" que puede 
econtrarse en \\ \texttt{http://www.ibiblio.org/pub/Linux/docs/LDP/system-admin-guide/translations/es/gasl.txt}.

Las marcas registradas son propiedad de sus dueños.

Se concede autorización para copiar, distribuir y/o modificar este documento
bajo los términos de la Licencia de documentación libre GNU (FDL), Versión 1.2; 
sin Secciones Invariantes, sin textos de Portada, y sin Textos de Contra Portada. 
Una copia de la licencia se encuentra incluída en la sección "GNU Free Documentation License".


\end{document}

