%%% LaTeX Template: Article/Thesis/etc. with colored headings and special fonts
%%%
%%% Source: http://www.howtotex.com/

\documentclass[12pt]{article}


\usepackage{apuntes-estilo}
\usepackage{fancyhdr,lastpage}



\def\maketitle{

% Titulo 
 \makeatletter
 {\color{bl} \centering \huge \sc \textbf{
Archivos, tipos y atributos \\
\large \vspace*{-8pt} \color{black} Una introducción al a los permisos y propiedad de archivos en GNU/Linux
 \vspace*{8pt} }\par}
 \makeatother


% Autor
 \makeatletter
 {\centering \small 
 	Departamento de Ingeniería de Computadoras \\
 	Facultad de Informática - Universidad Nacional del Comahue \\
 	\vspace{20pt} }
 \makeatother

}

% Custom headers and footers
\fancyhf{} % clear all header and footer fields
\fancypagestyle{plain}{\fancyhf{}}
  	\pagestyle{fancy}
 	\lhead{\footnotesize Archivos, tipos y atributos - Departamento de Ingeniería de Computadoras}
 	\rhead{\footnotesize \thepage\ }	% "Page 1 of 2"

\def\ti#1#2{\texttt{#1} & #2 \\ }



\begin{document}

\thispagestyle{empty}
\maketitle
\setlength{\parindent}{0pt}


\section*{Introducción}
Este documento describe los conceptos básicos sobre propiedades de archivos, 
incluyendo tipos y permisos de archivos, propietarios y grupos. 

\section*{Tipos de archivos}

Un archivo es una secuencia de bytes (un byte es un pequeño trozo de 
información, normalmente compuesto por ocho bits. Para nuestros propósitos
durante este texto, un byte es equivalente a un caracter). El sistema no impone 
estructura alguna a los archivos, ni asigna significado a su contenido; el 
significado de los bytes depende únicamente de los programas que 
interpretan el archivo. 

Vimos en apuntes anteriores que los archivos se organizan dentro de sistemas de 
archivos (ext3, ext4,NTFS, etc). Estos sistemas de archivos son el modo de 
ordenar los archivos dentro de los distintos medios de almacenamiento (discos
rígidos, de estado sólido, memorias flash, etc). 

En general, la mayoría de los sistemas de archivos de sistemas de tipo UNIX
definen seis tipos de archivos:

\begin{itemize}
\item Archivos regulares (Ej. un archivo de texto, un pdf son archivos regulares)
\item Directorios
\item Links simbólicos
\item Dispositivos (de bloque y caracter) 
\item UNIX Sockets  
\item Pipes 
\end{itemize}

Limitaremos nuestro estudio a los tres primeros tipos de archivos. 


\subsection*{Archivos regulares}
No son otra cosa que una bolsa de bytes: UNIX no impone ninguna estructura a
su contenid. Los archivos de texto, de datos, programas ejecutables, bibliotecas
compartidas, son todos archivos regulares.

La creación de archivos regulares sucede a través de distintos programas de 
aplicación, por ejemplo cuando utilizamos un sofware de ofimática, cuando 
descargamos archivos del mail, a través de un editor de texto, etc. El formato 
inerno de archivos regulares dependerá entonces de la aplicación que lo creó. 

La eliminación de archivos regulares sucede a través del comando \texttt{rm}. 

\subsection*{Directorios}
Los directorios son archivos que contienen referencias a otros archivos. 
Podemos crear directorios a través del comando \texttt{mkdir}, y eliminarlos
a tevés del comando \texttt{rmdir} si está vacío, o \texttt{rm -r} si tiene 
contenido. 

La estructura de un archivo de tipo directorio dependerá del sistema de 
archivos que usemos en particular. En general las entradas ``.'' y ``..''
se refieren al directorio actual o su padre respectivamente.  

\fcolorbox{black}{grey}{
\parbox[t]{1.0\linewidth}{ \vspace*{0.4cm}
{\bf Ejemplo} \\
\texttt{ 
\$ mkdir dir1 \\
\$ cd dir1\/ \\
\$ pwd \\
/home/lechnerm/test/dir1 \\
\$ touch arch\_dir1.txt  \\
\$ ls -l . \\
total 0 \\
-rw-r--r-- 1 lechnerm lechnerm 0 abr  4 20:06 arch\_dir1.txt \\
\$ mkdir dir2  \\
\$ cd dir2/ \\
\$ pwd \\
/home/lechnerm/test/dir1/dir2 \\
\$ touch arch\_dir2.txt \\
\$ ls -la . \\
total 8 \\
drwxr-xr-x 2 lechnerm lechnerm 4096 abr  4 20:07 . \\
drwxr-xr-x 3 lechnerm lechnerm 4096 abr  4 20:07 .. \\
-rw-r--r-- 1 lechnerm lechnerm    0 abr  4 20:07 arch\_dir2.txt \\
\$ ls -la .. \\
total 12 \\
drwxr-xr-x 3 lechnerm lechnerm 4096 abr  4 20:07 . \\
drwxr-xr-x 3 lechnerm lechnerm 4096 abr  4 20:06 .. \\
-rw-r--r-- 1 lechnerm lechnerm    0 abr  4 20:06 arch\_dir1.txt \\
drwxr-xr-x 2 lechnerm lechnerm 4096 abr  4 20:07 dir2 \\
\$ 
}

Nota: el comando \texttt{touch} crea un archivo regular vacío (o bytes). 
\vspace*{0.4cm} } }


Los \textit{nombres de arcdhivos} contenidos dentro de un directorio, se encuentran 
almacenados en el directorio y no en el archivo en sí mismo. Esto permite, entre otras
cosas, que un mismo archivo tenga varios nombres distintos \textit{dentro del mismo 
sistema de archivos}. Cada nombre es sólo una referencia, el contenido del archivo 
será el mismo cualquiera sea el nombre que usemos para accederlo. Esto es conocido
como ``hard link''. No debe confundirse un \textit{hard link} con una copia de archivo
por ejemplo creada a través del comando \texttt{cp}. 

Un \textit{hard link} se crea a  través del comando \texttt{ln} y se elmina igual 
que cualquier otro archivo regular, atrvés de el comando \texttt{rm}. 

\fcolorbox{black}{grey}{
\parbox[t]{1.0\linewidth}{ \vspace*{0.4cm}
{\bf Ejemplo} \\
\texttt{
\$ ls lunes.txt  \\
ls: no se puede acceder a lunes.txt: No existe el fichero o el directorio \\
\$ echo Hoy es lunes  \textgreater  lunes.txt  \\
\$ cat lunes.txt  \\
Hoy es lunes \\
\$ ls martes.txt \\
ls: no se puede acceder a martes.txt: No existe el fichero o el directorio \\
\$ ln lunes.txt martes.txt \\
\$ cat martes.txt  \\
Hoy es lunes \\
\$ echo Hoy es martes \textgreater \textgreater  martes.txt  \\
\$ cat lunes.txt  \\
Hoy es lunes \\
Hoy es martes \\
\$ rm lunes.txt  \\
\$ ls lunes.txt \\
ls: no se puede acceder a lunes.txt: No existe el fichero o el directorio \\
\$ cat martes.txt  \\
Hoy es lunes \\
Hoy es martes \\
\$  
}
\vspace*{0.4cm} } }


El sistema de archivos mantendrá la cuenta de los ``hard links'' que existen 
a un mismo archivo, decrementando el valor cada vez que se elimina una de las 
referencias, por ejemplo a través del comando \texttt{rm}. Sólo se liberará el espacio 
utilizado por el archivo cuando éste sea la última referencia. Podemos ver el 
número de referencias a un archivo con el comando \texttt{stat}.  

\fcolorbox{black}{grey}{
\parbox[t]{1.0\linewidth}{ \vspace*{0.4cm}
{\bf Ejemplo} \\
\texttt{
\$ cat martes.txt  \\
Hoy es lunes \\
Hoy es martes \\
\$ stat martes.txt  \\
  Fichero: «martes.txt» \\
  Tamaño: 27        	Bloques: 8          Bloque E/S: 4096   fichero regular \\
Dispositivo: fe01h/65025d	Nodo-i: 5638106    \textbf{Enlaces: 1} \\
Acceso: (0644/-rw-r--r--)  Uid: ( 1000/lechnerm)   Gid: ( 1000/lechnerm) \\
      Acceso: 2014-04-05 15:22:18.812061851 -0300 \\
Modificación: 2014-04-05 15:21:12.881692767 -0300 \\
      Cambio: 2014-04-05 15:22:13.716187888 -0300 \\
    Creación: - \\
\$ ln martes.txt lunes.txt  \\
\$ stat martes.txt  \\
  Fichero: «martes.txt» \\
  Tamaño: 27        	Bloques: 8          Bloque E/S: 4096   fichero regular \\
Dispositivo: fe01h/65025d	Nodo-i: 5638106    \textbf{Enlaces: 2} \\
Acceso: (0644/-rw-r--r--)  Uid: ( 1000/lechnerm)   Gid: ( 1000/lechnerm) \\
      Acceso: 2014-04-05 16:03:39.991411489 -0300 \\
Modificación: 2014-04-05 15:21:12.881692767 -0300 \\
      Cambio: 2014-04-05 16:03:36.739495849 -0300 \\
    Creación: - \\
\$ stat lunes.txt  \\
  Fichero: «lunes.txt» \\
  Tamaño: 27        	Bloques: 8          Bloque E/S: 4096   fichero regular \\
Dispositivo: fe01h/65025d	Nodo-i: 5638106     \textbf{Enlaces: 2} \\
Acceso: (0644/-rw-r--r--)  Uid: ( 1000/lechnerm)   Gid: ( 1000/lechnerm) \\
      Acceso: 2014-04-05 16:03:39.991411489 -0300 \\
Modificación: 2014-04-05 15:21:12.881692767 -0300 \\
      Cambio: 2014-04-05 16:03:36.739495849 -0300 \\
    Creación: - \\
\$ rm martes.txt  \\
\$ stat lunes.txt  \\
  Fichero: «lunes.txt» \\
  Tamaño: 27        	Bloques: 8          Bloque E/S: 4096   fichero regular \\
Dispositivo: fe01h/65025d	Nodo-i: 5638106     \textbf{Enlaces: 1} \\
Acceso: (0644/-rw-r--r--)  Uid: ( 1000/lechnerm)   Gid: ( 1000/lechnerm) \\
      Acceso: 2014-04-05 16:03:39.991411489 -0300 \\
Modificación: 2014-04-05 15:21:12.881692767 -0300 \\
      Cambio: 2014-04-05 16:04:09.702640844 -0300 \\
    Creación: - \\
\$  \\
}
\vspace*{0.4cm} } }

Es importante entender que los ``hard link'' no son un tipo de archivos especial. 
Simplemente el sistema de arhcivos permite crear múltiples referencias al 
mismo conjunto de bytes. Tanto el contenido como todos las propiedades del archivo
(propiedad, permisos, tiempos de acceso, etc). son compartidos entre todos los 
``hard link''.  

\subsection*{Links simbólicos}
Un \textit{link simbólcio} o ``soft link'' apunta a un nombre de archivo
(podría pensarse similar a los ``accesos directos'' de un sistema Windows).



\end{document}

