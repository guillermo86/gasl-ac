%%% LaTeX Template: Article/Thesis/etc. with colored headings and special fonts
%%%
%%% Source: http://www.howtotex.com/

\documentclass[12pt]{article}


\usepackage{apuntes-estilo}
\usepackage{fancyhdr,lastpage}



\def\maketitle{

% Titulo 
 \makeatletter
 {\color{bl} \centering \huge \sc \textbf{
Introducción a la Administración de Sistemas GNU/Linux \\
% \large \vspace*{-8pt} \color{black} Vim el Editor de Seis Billones de Dólares
 \vspace*{8pt} }\par}
 \makeatother


% Autor
 \makeatletter
 {\centering \small 
 	Departamento de Ingeniería de Computadoras \\
 	Facultad de Informática - Universidad Nacional del Comahue \\
 	\vspace{20pt} }
 \makeatother

}

% Custom headers and footers
\fancyhf{} % clear all header and footer fields
\fancypagestyle{plain}{\fancyhf{}}
  	\pagestyle{fancy}
 	\lhead{\footnotesize Introducción a la Administración de Sistemas GNU/Linux - Departamento de Ingeniería de Computadoras}
 	\rhead{\footnotesize \thepage\ }	% "Page 1 of 2"

\def\ti#1#2{\texttt{#1} & #2 \\ }



\begin{document}

\thispagestyle{empty}
\maketitle
\setlength{\parindent}{0pt}




\section{Introducción}

En este conjunto de tutoriales, y apuntes de cátedra,
se describen los aspectos del uso de GNU/Linux relativos a la administración del
sistema. Está destinado a personas con pocos conocimientos en la
administración del sistema (aquellos que se preguntan \textit{¿Qué es esto?}), pero que ya dominan
al menos los conceptos básicos sobre la utilización normal del mismo. Este
manual tampoco explica cómo instalar GNU/Linux; dicho tema está desarrollado en
el documento \textit{Instalación y Primeros Pasos}. En futuras secciones encontrará información
adicional sobre los manuales existentes para sistemas GNU/Linux.  

La administración de sistemas es el conjunto de tareas necesarias para
mantener un computador en buenas condiciones de uso (\textit{utilizable} para el resto
de los usuarios). Esto incluye actividades
tales como realizar copias de seguridad (y restaurarlas en caso necesario),
instalar nuevos programas, crear cuentas para los usuarios,
verificar la integridad de los sistemas de archivos, etc. Si un ordenador
fuese, por ejemplo, una casa, la administración del sistema podría ser comparada
con el 
mantenimiento hogareño, e incluiría la limpieza, la reparación de ventanas rotas, y otras
tareas similares.  

\fcolorbox{black}{grey}{
\parbox[t]{1.0\linewidth}{ \vspace*{0.4cm}
{\bf
Un administrador de sistemas GNU/Linux, generalmente, realiza alguna
o varias de las siguientes tareas esenciales : 

\begin{itemize}
	\item administrar cuentas de usuarios,
	\item agregar, configurar o quitar hardware,
	\item realizar copias de respaldo,
	\item instalar y configurar programas, 
	\item monitorear el sistema en busca de anomalias o mejora de performance,
	\item mantener la documentacion
	\item realizar tareas de seguridad informática,
	\item asistir a los usuarios,
	\item automatizar tareas, etc.
\end{itemize}
}
\vspace*{0.4cm} } }

La estructura de estos documentos permite utilizar muchos de los tutoriales y apuntes
de manera independiente, por lo que si necesita información relacionada, por ejemplo,
con backups, 
puede leer sólo el apunte que hace referencia a este tema. 
Sin embargo, estos apuntes son principalmente tutoriales,
y pueden ser leído secuencialmente, como un todo.

Además, es importante recalcar que estos apuntes no fueron
pensados para ser utilizado de forma aislada. También
es importante, para los administradores, el resto de la documentación
para sistemas GNU/Linux. Después de todo, un administrador de sistemas es sólo
un usuario con privilegios y obligaciones especiales. Un recurso muy útil son
las páginas de manual (también llamadas páginas man), las cuales deben ser
consultadas siempre que un comando no le sea familiar.  

Si bien estos documentos están centrados en GNU/Linux, un principio general de la
misma es el de procurar que también se puedan utilizar con otros sistemas
operativos basados en UNIX. Desafortunadamente, existen en
general versiones de UNIX diferentes, y en
particular existen diferencias en cuanto a la administración del sistema. 
Por lo que existen pocas esperanzas de que
cubra todas las variantes. Incluso cubrir todas las posibilidades para GNU/Linux
es difícil debido a la naturaleza de su desarrollo.  

No existe una distribución oficial de GNU/Linux, por lo que diferentes
personas tienen diferentes configuraciones, y muchas tienen una configuración
que ellos mismos realizaron. Esta documentación no está orientada a una distribución de
GNU/Linux en particular, ya que las distintas distribuciones varían
considerablemente entre sí. Por ello, siempre que sea posible se intenta hacer
notar las diferencias y desarrollar alternativas.

Se ha procurado describir cómo funciona cada aspecto del sistema, en vez
de limitarse a listar \textit{cinco pasos fáciles} para cada tarea. Esto significa que
existe
mucha información en estos documentos que puede no ser necesaria para todos, por lo
que dichas partes están marcadas especialmente y pueden ser
ignoradas si se está utilizando un sistema pre-configurado. Leyendo toda la documentacion
aumentará, naturalmente, la comprensión del funcionamiento del sistema, y se
podrá lograr que la utilización y la administración del sistema sea más
productiva (y agradable).

Un punto particular que debe aclararse es que no se han desarrollado en
profundidad muchos temas que se encuentran bien documentados en otros manuales
de libre distribución. Esto es aplicable especialmente a documentación de
programas concretos, como por ejemplo, todos los detalles de utilización del
comando \texttt{\textbf{mkfs}}. Tan sólo se describe el propósito del programa,
y como mucho, su utilización en la medida en que sea necesario para lograr el
propósito de estos documentos. Puede encontrar información adicional en otros
manuales libres. Normalmente, toda la documentación a la que se hace referencia
es parte del conjunto completo de documentación de GNU/Linux.  



\section{Visión general de un sistema GNU/Linux}

En esta sección se proporciona una visión
general de un sistema GNU/Linux.  En primer lugar se describen los principales
servicios que ofrece el sistema operativo.  A continuación, se explican con una
considerable falta de detalle los programas que implementan dichos servicios. El
propósito de este capítulo es hacer posible la comprensión del sistema en su
conjunto, por lo cual cada parte en concreto se encuentra descrita en
profundidad en capítulos posteriores.  


\subsection{Las diferentes partes de un sistema operativo}

 Un sistema operativo tipo UNIX consiste en un
\textit{núcleo} (o Kernel) y algunos programas de sistema. Existen
también diversos \textit{programas de aplicación} con los que
podemos trabajar.  El núcleo es el corazón del sistema operativo: Mantiene el
control de los archivos sobre el disco, inicia los programas y los ejecuta de
forma concurrente, asigna memoria y otros recursos a los distintos procesos,
recibe y envía paquetes desde y hacia la red, etc. El núcleo hace muy poco por
si solo, pero proporciona las herramientas básicas con las que se pueden
construir los demás servicios.
\footnote{De hecho, a menudo es considerado erróneamente como el sistema
operativo en sí, aunque no lo es. Un sistema operativo proporciona muchos más
servicios que el núcleo	por sí mismo.}
Además, evita que se pueda acceder al hardware
directamente, forzando a todos a utilizar las herramientas provistas. Esta
manera de trabajar del núcleo otorga cierta protección a los usuarios entre sí.
Las herramientas del núcleo se utilizan a través de las llamadas al sistema. Se
puede encontrar información adicional sobre este tema consultando la sección 2
de este manual.

 Los programas de sistema utilizan las herramientas provistas por el
núcleo para implementar varios servicios requeridos en un sistema operativo. Los
programas de sistema (y todos los demás programas), se ejecutan 'por encima del
núcleo', en lo que se denomina modo usuario. La diferencia entre los programas
de aplicación y los de sistema es su finalidad: las aplicaciones tienen el
propósito de realizar tareas útiles a los usuarios (o de jugar, si se tratara de
un juego), mientras que los programas de sistema son necesarios para que el
sistema funcione. Un procesador de textos es una aplicación;
\texttt{\textbf{mount}} es un programa de sistema. La diferencia a menudo es
confusa, y de cualquier manera, es solo importante para los categorizadores
compulsivos.  

 Un sistema operativo también puede contener compiladores y sus
correspondientes librerías (GCC y la librería de C en particular para
GNU/Linux), aunque no todos los compiladores de todos los lenguajes de
programación son necesariamente parte del sistema operativo. También puede haber
documentación, y en algunas ocasiones juegos. Tradicionalmente, el sistema
operativo está definido por el contenido de los discos o cintas de instalación;
con GNU/Linux esta definición no puede aplicarse, debido a que se encuentra
repartido sobre distintos sitios FTP del mundo.  



\subsection{Partes importantes del núcleo}

El núcleo de un sistema GNU/Linux consta de varias partes importantes:
gestión de procesos, gestión de memoria,  controladores para dispositivos de
hardware, controladores para sistemas de archivos, gestión de la red, y otras
partes varias. La 
% <xref linkend="kerneloverview"/> 
muestra algunas de éstas partes.

%<figure id="kerneloverview" float="1"> 
Partes más importantes del núcleo
de GNU/Linux
%<graphic fileref="figuras/overview-kernel.png"/> </figure>

 Probablemente las partes más importantes del núcleo (nada funcionaría sin
ellas) son la gestión de memoria y la gestión de procesos. El gestor de memoria
se encarga de asignar áreas de memoria y de espacio de intercambio a los
procesos, partes del núcleo, y también al buffer caché. El gestor de
procesos crea nuevos procesos e implementa la multitarea (intercambiando
los procesos activos en el procesador).

A más bajo nivel, el núcleo contiene un controlador de dispositivo de
hardware para cada tipo de hardware que soporta. Debido a que el mundo se
encuentra lleno de diferentes tipos de hardware, el número de controladores es
grande. Existen frecuentemente, muchas piezas similares de hardware que difieren
en cómo son controladas por el software. Esta singularidad hace posible tener
clases generales de controladores que soportan operaciones similares; cada
miembro de la clase tiene la misma interfaz de cara al resto del núcleo pero
difiere de los demás miembros en la forma de implementar las operaciones. Por
ejemplo, todos los controladores de disco son parecidos para el resto del
núcleo, P.ej., todos tienen operaciones como \textit{iniciar la unidad}, 
\textit{leer el sector n}, y \textit{escribir en el sector n}.

 Algunos servicios de software provistos por el núcleo tienen propiedades
similares, y pueden de esta manera englobarse dentro de clases. Por ejemplo, los
diferentes protocolos de red fueron englobados dentro de una interfaz de
programación, la librería de socket BSD. Otro ejemplo es la capa del
\textit{sistema de archivos virtual} (VFS) que abstrae las
operaciones de los sistemas de archivos de sus implementaciones. Cada tipo de
sistema de archivos provee una implementación de cada operación. Cuando alguna
entidad intenta utilizar un sistema de archivos, la petición se realiza a través
del VFS, el cual la encamina al controlador del sistema de archivos correcto.




\subsection{Servicios principales en un sistema UNIX}

 En esta sección se describen algunos de los servicios más importantes en
UNIX, pero sin mucho detalle. Se describirán más profundamente en capítulos
posteriores.  


\subsubsection{\texttt{\textbf{init}}}

 El servicio individual más importante en un sistema UNIX es provisto por
\texttt{\textbf{init}}. \texttt{\textbf{init}} es el primer proceso que se
inicia en todo sistema UNIX, siendo la última acción que el núcleo realiza al
arrancar.  Cuando init comienza su ejecución, continúa con el proceso de
arranque del sistema, realizando varias tareas de inicio (chequear y montar
sistemas de archivos, iniciar demonios, etc.).  

 La lista exacta de cosas que \texttt{\textbf{init}} realiza depende del sistema tipo UNIX
con el que estemos trabajando; existen varios para elegir. \texttt{\textbf{init}} normalmente
proporciona el concepto de \textit{modo de usuario individual (single user mode)}, en el cual nadie puede
iniciar una sesión y root utiliza un intérprete de comandos en la consola; el
modo usual es llamado \textit{modo multiusuario (multiuser mode)}.
Algunos sistemas UNIX generalizan esto
como \textit{niveles de ejecución (run levels)}. Así, los modos individual y
multiusuario son considerados dos niveles de ejecución, y pueden existir otros
niveles adicionales para, por ejemplo, ejecutar X-Windows en la consola.


 GNU/Linux permite tener hasta 10 \textit{niveles de ejecución (runlevels)} distintos, 0-9,
pero normalmente solo algunos de estos niveles están definidos por defecto. El
nivel de ejecución 0 se define como \textit{sistema detenido (system halt)}. El nivel
de ejecución 1 se define como \textit{modo de usuario individual (single user mode)}.
El nivel de ejecución 6 se define como \textit{reinicio del sistema (system reboot)}.
Los niveles de ejecución restantes dependen de como la distribución particular
de GNU/Linux los haya definido, y varían significativamente entre
distribuciones. Observando el contenido del archivo
\texttt{/etc/inittab} podemos hacernos una idea de los niveles de
ejecución preestablecidos en nuestro sistema y de como se encuentran definidos.


 En el funcionamiento normal, \texttt{\textbf{init}} se asegura de que
\texttt{\textbf{getty}} se encuentre trabajando para permitir que los usuarios puedan iniciar una
sesión, y también se encarga de adoptar procesos huérfanos (aquellos cuyo
proceso padre murió; en UNIX \textit{todos} los procesos \textit{deben} estar
en un árbol individual, y por esta razón los procesos huérfanos deben ser
adoptados).  

 Al cerrar el sistema, es \texttt{\textbf{init}} quien se encarga de
matar todos los procesos restantes, desmontar todos los sistemas de archivos, y
por último detener el procesador, además de cualquier otra cosa que haya sido
configurado para hacer.  




\subsubsection{ Inicio de sesiones desde terminales}

 El inicio de sesiones desde terminales (a través de líneas serie) y la
consola (cuando no se está ejecutando X-Windows) es suministrado por el programa
\texttt{\textbf{getty}}. \texttt{\textbf{init}} inicia una instancia
independiente de \texttt{\textbf{getty}} por cada terminal en el que está permitido iniciar
sesiones. \texttt{\textbf{Getty}} lee el nombre de usuario y ejecuta el
programa login, el cual se encarga de leer la password. Si el nombre de usuario
y la password son correctas, \texttt{\textbf{login}} ejecuta el intérprete de
comandos.  Al finalizar el intérprete de comandos (en el caso en que, por
ejemplo, el usuario finaliza su sesión; o cuando \texttt{\textbf{login}} finaliza debido a que no
concuerdan el nombre de usuario y la password), \texttt{\textbf{init}} se
entera de este suceso e inicia una nueva instancia de \texttt{\textbf{getty}}.
El núcleo no tiene noción sobre los inicios de sesiones, esto es gestionado
totalmente por los \textit{programas del sistema}.  



\subsubsection{ Syslog}

 El núcleo y muchos \textit{programas de sistema} producen
mensajes de error, de advertencia y de otros tipos. La mayoría de las veces, es
importante que puedan ser visualizados mas tarde, o tal vez mucho después, por
lo que tales mensajes deben guardarse en un archivo. El programa que realiza
esta tarea es \texttt{\textbf{syslog}}. Syslog puede ser configurado para
ordenar los mensajes en diferentes archivos, de acuerdo a quien lo emite o al
grado de importancia.  Por ejemplo, los mensajes del núcleo son frecuentemente
dirigidos a un archivo separado de los demás, debido a que son más importantes,
y necesitan ser leídos regularmente para detectar problemas.  



\subsubsection{ Ejecución periódica de comandos: \texttt{\textbf{cron}} y
\texttt{\textbf{at}}}

 Los administradores de sistemas y los usuarios, a menudo necesitan
ejecutar comandos periódicamente. Como ejemplo, supongamos que el administrador
del sistema desea ejecutar un comando que elimine los archivos más antiguos de
los directorios con archivos temporales (\texttt{/tmp} y
\texttt{/var/tmp}) para evitar así que el disco se llene, debido a
que no todos los programas eliminan correctamente los archivos temporales que
ellos mismos generan.  

 El servicio \texttt{\textbf{cron}} se configura para que realice la
tarea anterior. Cada usuario tiene un archivo \texttt{crontab}, en
el cual se listan los comandos que se desea ejecutar y la fecha y hora de
ejecución. El servicio \texttt{\textbf{cron}} se encarga con precisión de
iniciar cada comando, a la fecha y hora adecuada de acuerdo a lo especificado en
cada archivo crontab.  

 El servicio \texttt{\textbf{at}} es similar a \texttt{\textbf{cron}},
pero este se inicia únicamente una vez: el comando es ejecutado a la hora
especificada, pero esta ejecución no vuelve a repetirse.  

 Se puede encontrar información adicional sobre cron(1), crontab(5), at(1)
y atd(8) en las páginas de manual.  



\subsubsection{ Interfaz gráfica de usuario (GUI)}

 UNIX y GNU/Linux no incorporan la interfaz gráfica de usuario dentro del
núcleo; en su lugar, es implementada por programas a nivel de usuario. Esto se
aplica tanto a entornos gráficos como al modo texto.  

 Esta disposición hace que el sistema sea más flexible, pero tiene la
desventaja de que, al ser simple implementar una interfaz de usuario diferente
para cada programa, dificulta el aprendizaje del sistema.  

 El entorno gráfico principalmente utilizado con GNU/Linux se llama
Sistema X-Windows (X para abreviar). X tampoco implementa por sí mismo una
interfaz de usuario, sino solo un sistema de ventanas. Es decir, las
herramientas base con las cuales se puede construir una interfaz gráfica de
usuario. Algunos administradores de ventanas populares son: fvwm, icewm,
blackbox y windowmaker. Existen también dos populares administradores de
escritorios: KDE y Gnome.  




\subsubsection{ Redes}

 Una red se construye al conectar dos o más ordenadores para que puedan
comunicarse entre sí. Los métodos actuales de conexión y comunicación son
ligeramente complicados, pero el resultado final es muy útil.  

 Los sistemas operativos UNIX tienen muchas características de red. La
mayoría de los servicios básicos (sistemas de archivos, impresión, copias de
seguridad, etc) pueden utilizarse a través de la red. Aprovechar estas
características puede ayudar a que la administración del sistema sea más fácil
debido a que permiten tener una administración centralizada, a la vez que
disfrutamos de los beneficios de la micro informática y la informática
distribuida, tales como costes más bajos y mejor tolerancia a fallos.  

 De cualquier modo, este libro sólo aborda superficialmente la teoría de
redes; Se puede encontrar información adicional sobre este tema en La Guía De
Administración De Redes con  Linux (\textit{Linux Network Administrators' Guide}
%/citetitle>
	http://www.tldp.org/LDP/nag2/index.html
%	http://www.tldp.org/LDP/nag2/index.html</ulink>), 
incluyendo una descripción básica de como
operan las redes.  



\subsubsection{ Inicio de sesiones a través de la red}

 Los inicios de sesión a través de la red funcionan de un modo un poco
diferente al inicio de sesiones normales. Existe una línea serie física separada
para cada terminal a través de la cual es posible iniciar sesión. Por cada
persona iniciando una sesión a través de la red existe una conexión de red
virtual, y puede haber cualquier número (no hay límite).  
\footnote{Al menos puede haber muchas. Dado que el ancho de banda es un
recurso escaso, existe aún en la práctica algún límite al
número de inicios de sesión concurrentes a través de una conexión
de red. }

Por lo tanto, no es
posible ejecutar \texttt{\textbf{getty}} por separado por cada conexión virtual posible. Existen
también varias maneras diferentes de iniciar una sesión a través de la red, las
principales en redes TCP/IP son \texttt{\textbf{telnet}} y \texttt{\textbf{rlogin}}.
\footnote{Hoy en día muchos administradores de sistemas Linux consideran
que \texttt{\textbf{telnet}} y \texttt{\textbf{rlogin }} son inseguros y
prefieren \texttt{\textbf{ssh }}, el \textit{intérprete de comandos
seguro} que 			encripta el tráfico en la red, haciendo
así bastante menos probable que usuarios malintencionados puedan
\textit{espiar} la 				conexión y obtener datos
sensibles como nombres de usuario y 				passwords. Está
altamente recomendado usar \texttt{\textbf{ssh}} 			en lugar
de \texttt{\textbf{telnet}} o \texttt{\textbf{rlogin}}.  }


 Los inicios de sesión a través de la red tienen, en vez de una cantidad
enorme de \texttt{\textbf{getty's}}, un servicio individual por tipo de inicio de sesión (\texttt{\textbf{telnet}} y \texttt{\textbf{rlogin}}
tienen servicios separados) que \textit{escucha} todos los intentos de inicio de
sesión entrantes. Cuando el servicio advierte un intento de inicio de sesión,
inicia una nueva instancia de si mismo para atender la petición individual; la
instancia original continúa atenta a otros posibles intentos. La nueva instancia
trabaja de manera similar a \texttt{\textbf{getty}}.  




\subsubsection{ Sistemas de archivos de red (NFS)}  Una de las cosas
más útiles que se pueden hacer con los servicios de red es compartir archivos a
través de un \textit{sistema de archivos de red}. El más utilizado
normalmente para compartir archivos se llama \textit{Network File
System}, o \textit{NFS}, desarrollado por Sun
Microsystems.  

 Con un sistema de archivos de red, cualquier operación sobre un archivo
realizada por un programa en una máquina es enviada a través de la red a otra
máquina. Se \textit{engaña} al programa, haciéndole creer que todos los archivos en el
ordenador remoto se encuentran de hecho en el ordenador en el que el programa se
está ejecutando. Con esta manera de trabajar, compartir información es
extremadamente simple, ya que no se requieren modificaciones en el programa.


 Otra manera muy popular de compartir archivos es a través de Samba
(http://www.samba.org). Este protocolo
(llamado SMB) permite compartir archivos con máquinas Windows a través del
Entorno de Red. También permite compartir impresoras.  



\subsubsection{ Correo}

 El correo electrónico es el método más popularmente utilizado para
comunicarse a través del ordenador. Una carta electrónica se almacena en un
archivo con un formato especial, y se utilizan programas de correo especiales
para enviar y leer las cartas.  

 Cada usuario tiene un \textit{buzón de correo entrante} (un
archivo con formato especial), en donde se almacena todo el correo nuevo. Cuando
alguien  envía un correo, el programa de correo localiza el buzón del
destinatario y agrega la carta al archivo de buzón de correo entrante. Si el
buzón del destinatario se encuentra en otra máquina, la carta es enviada allí,
donde se traslada al buzón de correo como corresponda.  

 El sistema de correo se compone de muchos programas. El transporte del
correo a buzones locales o remotos es realizado por un programa: \textit{el
agente de transporte de correo} o \textit{MTA}.
(\texttt{\textbf{Sendmail}} y \texttt{\textbf{Smail}} son dos ejemplos de
esto), mientras que existe un sin número de programas muy variados que los
usuarios utilizan para leer y escribir correos (\textit{Estos son conocidos
como agentes de usuario de correo }o \textit{MUA},
\texttt{\textbf{Pine}} y \texttt{\textbf{Elm}} son ejemplos de esto). Los
archivos de buzones de correo están usualmente ubicados en
\texttt{/var/spool/mail}.  




\subsubsection{ Impresión}

 Solo una persona puede utilizar la impresora en un momento dado, pero
sería antieconómico no compartir impresoras entre los usuarios. La impresora es
por lo tanto administrada por software que implementa una cola de impresión:
todos los trabajos de impresión son colocados dentro de la cola, y una vez que
la impresora termina de imprimir una trabajo, el siguiente es enviado a la
impresora automáticamente. Esto alivia al usuario de la organización de la cola
de impresión y de luchar por el control de la impresora.  

 El software de la cola de impresión también coloca los trabajos de
impresión en disco, es decir, el texto a imprimir es mantenido en un archivo
mientras que el trabajo se encuentre en la cola. Esto permite a los programas de
aplicación entregar rápidamente los trabajos a imprimir al software que
administra la \textit{cola de impresión}; así, las aplicaciones no
tienen que esperar a que el trabajo (en inglés \textit{job}) esté de hecho impreso para
poder continuar su ejecución. Esta forma de trabajar es realmente cómoda, ya que
permite enviar a imprimir una versión de un trabajo y no tener que esperar a que
ésta sea impresa antes de poder hacer una versión nueva completamente revisada.



\fcolorbox{black}{grey}{
\parbox[t]{1.0\linewidth}{ \vspace*{0.4cm}
El servicio de impresión mas utilizado en los sistemas GNU/Linux
es \textbf{CUPS}. \textbf{CUPS} es un sistema de impresión open source
desarrollado por Apple(c) para sistemas UNIX \textbf{\texttt{http://www.cups.org}}
\vspace*{0.4cm} } }




\subsubsection{ La distribución del sistema de archivos}

 El sistema de archivos está dividido en muchas partes; normalmente en las
líneas de un sistema de archivos raíz con \texttt{/bin},
\texttt{/lib}, \texttt{/etc}, \texttt{/dev},
y otros pocos directorios; un sistema de archivos \texttt{/usr} con programas y datos que
no tendrán cambios; un sistema de archivos \texttt{/var} con datos que pueden cambiar
(como los archivos de log); y un sistema de archivos \texttt{/home}
para todos los archivos personales de los usuarios. Dependiendo de la
configuración del hardware y de las decisiones del administrador del sistema, la
división puede llegar a ser diferente; a pesar de esto, y aunque la división es
aconsejable, es también posible distribuir todos los archivos en un solo sistema
de archivos.  

 En el 
% <xref linkend="dir-tree-overview"/> 
se describe la distribución del sistema de archivos con
algo de detalle; el documento \textit{Estándar de la Jerarquía del Sistema de Archivos
de Linux} cubre este tema más en profundidad.
\footnote{http://www.pathname.com/fhs/}

 


\end{document}
