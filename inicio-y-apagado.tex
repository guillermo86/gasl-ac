
\documentclass[12pt]{article}

\usepackage{apuntes-estilo}
\usepackage{fancyhdr,lastpage}



\def\maketitle{

% Titulo 
 \makeatletter
 {\color{bl} \centering \huge \sc \textbf{
 Inicio y Apagado de uns Sistema GNU/Linux \\ 
%  \large \vspace*{-8pt} \color{black} Vim el Editor de Seis Billones de Dólares
 \vspace*{8pt} }\par}
 \makeatother


% Autor
 \makeatletter
 {\centering \small 
 	Departamento de Ingeniería de Computadoras \\
 	Facultad de Informática - Universidad Nacional del Comahue \\
 	\vspace{20pt} }
 \makeatother

}

% Custom headers and footers
\fancyhf{} % clear all header and footer fields
\fancypagestyle{plain}{\fancyhf{}}
  	\pagestyle{fancy}
 	\lhead{\footnotesize Inicio y Apagado - Departamento de Ingeniería de Computadoras}
 	\rhead{\footnotesize \thepage\ }	% "Page 1 of 2"

\def\ti#1#2{\texttt{#1} & #2 \\ }



\begin{document}

\thispagestyle{empty}
\maketitle
\setlength{\parindent}{0pt}



\section{Encendido y apagado}


 En esta sección se explica lo que sucede en un sistema Linux al momento
de encender o apagar la computadora, y de cómo debe hacerse apropiadamente. Si
no se sigue el procedimiento adecuado, los archivos se pueden dañar o perder.


\section{ Una introducción al proceso de inicio y finalización del
sistema}

Al acto de encender un computador y cargar su sistema operativo 
		\footnote{On early computers, it wasn't enough to merely
		turn on the computer, you had to manually load the operating
		system as well.  These new-fangled thing-a-ma-jigs do it all by
		themselves.} 
se le llama en Inglés \textit{booting}. El nombre viene de
la idea del computador iniciándose a sí mismo sin ayuda.  

Durante el inicio, o bootstrapping, lo primero que el computador carga en
memoria y ejecuta es un pequeño programa llamado cargador o \textit{bootstrap
loader}, el cual a su vez carga el sistema operativo en memoria e
inicia su ejecución. El cargador o bootstrap loader, se encuentra almacenado en
una dirección fija dentro de un disco duro o en un floppy. La razón de que
existan dos etapas en este proceso se debe a que el sistema operativo es grande
y complicado, pero la primera pieza de código que el computador carga debe ser
muy pequeña (unos cuantos cientos de bytes), para evitar que el firmware (su
almacenamiento) sea innecesariamente complicado.  

 Diferentes tipo de computadoras realizan la primera etapa de inicio
(bootstrapping) de manera distinta. Las computadoras de tipo PC'S (arquitectura
x86) cuentan con un programa (BIOS: Basic Input Output System) que lee el
cargador de arranque almacenado en el primer sector de un disco duro o un floppy
(sector de arranque o \textit{boot sector}). El cargador, al
ejecutarse, carga el sistema operativo almacenado en otro lugar del disco (o
ubicado en otro lugar).  

 El kernel Linux luego de que ha sido cargado en memoria y comienza a
ejecutarse, inicializa el hardware y los controladores de los dispositivos. Por
último ejecuta el programa \texttt{\textbf{INIT}}. \texttt{\textbf{INIT}} a su
vez inicia otros procesos que permite a los usuarios iniciar sesiones (log in) y
trabajar dentro del sistema. Los detalles de esta parte del inicio del sistema
serán descritos con mayor detalle en otro capítulo.  

 Al finalizar un sistema Linux (shutdown), se le ordena a cada proceso
terminar. Esto causa que los archivos sean cerrados y que se realicen otras
tareas que sirven para mantener el sistema en orden. A continuación se desmontan
los sistemas de archivos y las áreas de swap. Finalmente, se muestra un mensaje
en la consola indicando que el equipo puede apagarse. Si no se sigue el
procedimiento anterior pueden y sucederán cosas terribles. Lo mas problemático
que puede pasar, es que el buffer caché del sistema de archivos no sea escrito a
disco (o a su medio de almacenamiento correspondiente), lo que significa que los
datos contenidos en él se pierdan y que por lo tanto la información contenida en
el sistema de archivos del disco sea inconsistente y posiblemente inutilizable.
 


\section{ Una mirada más cercana al proceso de inicio}

 Puede iniciar su sistema Linux desde un disco duro o disquete. La sección
de instalación de la Guía GNU/Linux: Instalación y Primeros Pasos, se explica
cómo instalar el sistema operativo para arrancar de un modo u otro.  



 Al encender una PC, el BIOS efectúa varias pruebas al hardware para
asegurarse de que no existan problemas\footnote{El nombre de este chequeo es
\textit{power on self test}, or \textit{POST} for
short.}, y luego empieza realmente el proceso de inicio del sistema operativo. El BIOS
selecciona una unidad de disco y lee su primer sector. Típicamente, la unidad de
disco seleccionada es la unidad de disquetes (y funciona si existe un disquete
insertado booteable), de no ser así selecciona el primer disco duro, si existe
un disco en la computadora (de todas maneras el orden de selección de unidad de
discos es configurable en la mayoría de las PC's). Al primer sector de un disco
se le llama \textit{sector de arranque}, en inglés boot sector o
\textit{master boot sector} (MBR o sector de arranque maestro). Si
el disco contiene varias particiones, cada una de ellas tiene su propio sector
de arranque (primer sector de cada partición).  


 El sector de arranque (boot sector) contiene un pequeño programa lo
suficientemente pequeño para que quepa en un único sector, cuya responsabilidad
es leer el sistema operativo desde el disco, cargarlo en memoria y ponerlo en
ejecución. Cuando se inicia Linux desde un disquete, el sector de arranque
contiene un código que solamente lee los primeros cientos de bloques
(dependiendo del tamaño del kernel por supuesto) y los coloca en una ubicación
predeterminada en la memoria. En un disquete de inicio de Linux no existe un
sistema de archivos, el kernel está simplemente almacenado en sectores
consecutivos, lo que simplifica el proceso de arranque. Sin embargo es posible
también arrancar desde un disquete con un sistema de archivos utilizando LILO,
el cargador de Linux.  


 Al arrancar desde un disco duro, el código contenido en el sector de
arranque maestro (MBR o Master Boot Record), examina la tabla de particiones del
disco duro (esta tabla también se almacena en el MBR), identifica la partición
activa (aquella partición que ha sido designada como de inicio), lee el cargador
contenido en el sector de arranque (boot sector) a una localización en memoria y
lo ejecuta.  

 El código almacenado en el sector de arranque de la partición del disco
duro hace lo mismo que el contenido en el sector de arranque del diskette: lee
el kernel desde la partición del disco duro, lo coloca en memoria e inicia su
ejecución. Sin embargo, existen diferencias. No es muy útil mantener una
partición exclusiva para almacenar la imagen del kernel, por lo tanto el
cargador no se limita a leer de manera secuencial los sectores del disco, como
lo hace en el caso de un floppy, sino que en el caso de un disco duro debe
encontrar en qué lugar del sistema de archivos ha sido almacenado el kernel.
Existen varias maneras de resolver este problema, pero la más común es
utilizando LILO (consulte la documentación de LILO si desea conocer mayores
detalles de cómo LILO realiza esta tarea).  

LILO normalmente lee, coloca en memoria y comienza la ejecución del kernel
Linux predeterminado. Es posible configurar LILO para que sea capaz de iniciar
una de varias imágenes del kernel diferentes, o de otros sistemas operativos. De
esta manera, también es posible que el usuario seleccione con cuál de ellos
trabajar al momento de iniciar el sistema.  Se puede configurar LILO, para que
no inicie la carga del kernel predeterminado de manera inmediata y consulte cuál
kernel o sistema operativo debe iniciarse (esta configuración necesita que se
mantengan presionadas la teclas de \texttt{alt}, \texttt{shift},
or \texttt{ctrl} cuando LILO comienza su ejecución). Otra opción
consiste en configurarlo de manera que siempre consulte desde qué imagen o
sistema se va a iniciar, y que transcurrido cierto tiempo de no recibir ninguna
indicación, arranque desde el kernel predeterminado.  



 Con LILO es posible "pasarle" al kernel información a través de
argumentos en la línea de comandos (\textit{kernel command line
argument}), tecleando los argumentos después del nombre del kernel o
del sistema operativo que se desea utilizar.  


 Iniciar el sistema desde un disquete o desde un disco duro tienen
ventajas diferentes, aunque generalmente iniciar desde el disco duro es mas
agradable, ya que evita la incomodidad de lidiar con diskettes. También es más
rápido. Sin embargo, en un principio puede ser complejo para algunos usuarios
instalar el sistema para que arranque desde un disco duro. Por lo que, a veces
se instala de manera que inicie desde un disquete y después cuando el sistema
está instalado de manera adecuada y todo funciona correctamente se instala el
LILO para arrancar desde el disco duro.  

 Después de que el kernel ha sido leído del disco, cargado en la memoria y
puesto en ejecución, pasa lo siguiente: \begin{itemize}

	\item{

Se descomprime el kernel, ya que éste se almacena comprimido, al principio de la
imagen del kernel, se encuentra un pequeño programa que se encarga de esta
tarea.  }

	\item{

Si la máquina cuenta con una tarjeta super VGA que Linux reconoce y soporta
modos especiales de texto (tales como 100 columnas por 40 renglones), Linux
pregunta cuál es el modo que se desea usar. Al compilar de principio el kernel
se puede determinar de antemano que modo de video usar, de manera que Linux
nunca pregunte. También se puede determinar mediante la configuración de LILO o
por el comando rdev.  }

	\item{

Una vez realizado lo anterior, el kernel verifica qué otro hardware existe
dentro de la máquina (discos duros, unidades de diskettes, tarjetas de red,
adaptadores, etc...), y configura sus controladores de manera apropiada.
Conforme lo va haciendo, envía mensajes a la consola acerca de lo que se va
encontrando. Por ejemplo, al iniciar una computadora puede verse lo siguiente:

  
   


\colorbox{grey}{\parbox[t]{0.95\linewidth}{ \vspace*{0.5cm} {\tt 
LILO boot:
Loading linux.
Console: colour EGA+ 80x25, 8 virtual consoles
Serial driver version 3.94 with no serial options enabled
tty00 at 0x03f8 (irq = 4) is a 16450
tty01 at 0x02f8 (irq = 3) is a 16450
lp\_init: lp1 exists (0), using polling driver
Memory: 7332k/8192k available (300k kernel code, 384k reserved, 176k
data)
Floppy drive(s): fd0 is 1.44M, fd1 is 1.2M
Loopback device init
Warning WD8013 board not found at i/o = 280.
Math coprocessor using irq13 error reporting.
Partition check:
  hda: hda1 hda2 hda3
VFS: Mounted root (ext filesystem).
Linux version 0.99.pl9-1 (root@haven) 05/01/93 14:12:20
 } \vspace*{0.5cm} } } 




El idioma y la forma exacta del texto es diferente en cada sistema, en función
del hardware instalado, la versión de Linux que se está utilizando y como se ha
establecido la configuración del sistema.  }

	\item{
A continuación el kernel trata de montar el sistema de archivos raíz (root
filesystem, "/"). El punto de montaje se puede determinar al momento de compilar
el kernel o después mediante el uso de LILO o del comando
\texttt{\textbf{rdev}}. El tipo del sistema de archivos se detecta
automáticamente. Si la acción de montar el sistema de archivos raíz falla, por
ejemplo debido a que no recordó incluir el controlador del sistema de archivos
correspondiente dentro del kernel, el kernel entra en pánico y detiene la
actividad del sistema operativo (de todas maneras, no puede hacer más).  

 El sistema de archivos raíz se monta en modo de sólo lectura, lo anterior
se determina al momento de compilar el kernel, lo que permite verificar el
estado del sistema de archivos (filesystem check) al momento de ser montado. No
es una buena idea verificar la integridad de un sistema de archivos que es
montado en modo de lectura/escritura.  }

	\item{ El paso siguiente consiste en iniciar la  ejecución del
	programa \texttt{\textbf{init}} (almacenado en
	\texttt{/sbin/init}) en segundo plano (background). El
	programa \texttt{\textbf{INIT}} siempre será el proceso con ID 1. INIT
	realiza varias tareas relacionadas con el inicio del sistema. La manera
	exacta en que las realiza depende en cómo sea configurado. Puede leer el
	apunte sobre init para obtener mayor información al respecto. Al
	menos, INIT iniciará algunos demonios en segundo plano esenciales.
	}

	\item{ Acto seguido, \texttt{\textbf{init}} cambia a modo
	multiusuario y ejecuta un \texttt{\textbf{getty}} (get tty) para las
	consolas virtuales y la líneas seriales. \texttt{\textbf{Getty}} es el
	programa  que permite a las personas iniciar sesiones (log in) a través
	de las consolas virtuales y terminales conectadas vía puertos seriales.
	\texttt{\textbf{INIT}} puede también ejecutar otros programas
	dependiendo del modo en que esté configurado.  }

	\item{

Después de todas las etapas anteriores, el inicio de Linux está completo y el
sistema se encuentra activo de manera normal.  }

	\end{itemize} 



\section{ Más acerca de shutdown}

 Es sumamente importante comprender y seguir los pasos correctos al
momento de finalizar un sistema Linux. Si no realiza este procedimiento, sus
sistemas de archivos se verán probablemente perjudicados y los archivos pueden
quedar desordenados. Esto sucede porque Linux tiene un cache de disco que no
escribe a disco cada vez que se le solicita, sino que solamente lo hace a
intervalos. Esta manera de proceder mejora de manera significativa el desempeño
del sistema, pero al mismo tiempo significa que si se apaga el equipo de
imprevisto puede perder información que debería estar en sus sistemas de
archivos (y obviamente en sus discos). Esto último sucede porque la cache puede
contener una gran cantidad de datos que se pierden con el apagado, y lo que está
en el disco puede no ser un sistema de archivos totalmente funcional (debido a
que solamente parte de la información ha sido transcrita de la cache al disco
duro).  

 Una razón adicional para no desconectar directamente el sistema de la
energía (presionando por ej. el botón de apagado) es que en un ambiente
multitarea existen diversos procesos que se están ejecutando en segundo plano, y
desconectar la computadora en este momento puede ser desastroso. Utilizando el
procedimiento correcto para el apagado del equipo garantiza que todos los
procesos en segundo plano puedan guardar sus datos.  

 El comando para finalizar correctamente un sistema Linux es
\texttt{\textbf{shutdown}}. Se utiliza generalmente de una de dos maneras
diferentes: 

 Si Ud. es el único usuario del sistema, debe finalizar todos los
programas que estén en ejecución, finalizar todas las sesiones (log out) de
todas las consolas virtuales, e iniciar una sesión como usuario root (o mantener
la sesión si ya existe una, pero debe cambiar de directorio de trabajo al
directorio HOME de root, para evitar problemas al desmontarse los sistemas de
archivos). Finalmente ejecute el comando \texttt{\textbf{shutdown -h now}}. Si
desea postergar durante algún lapso el comando shutdown, reemplace
\texttt{now} con un signo + (mas) y un numero que indica minutos de
espera.  

 Alternativamente, si el sistema está siendo utilizado por muchos
usuarios, utilice el comando shutdown -h +time mensaje, donde time es el numero
de minutos en que se posterga la detención del sistema, y el mensaje es una
explicación breve de el porqué se está apagando el sistema.

\colorbox{grey}{\parbox[t]{0.95\linewidth}{ \vspace*{0.5cm} 
{\tt
\# shutdown -h +10 'We will install a new 
disk.  System should be back on-line in three hours.' \\
\#
} \vspace*{0.5cm} } } 

El ejemplo advierte a todos los usuarios que el sistema se apagará en diez
minutos, y que sería mejor que se desconectaran o se arriesgan a perder la
información con la que están trabajando. La advertencia se muestra en cada
terminal donde existe un usuario conectado, incluyendo las xterm (emuladores de
terminales para el sistema X Window).



\colorbox{grey}{\parbox[t]{0.95\linewidth}{ \vspace*{0.5cm} 
{\tt  Broadcast message from root (ttyp0) Wed Aug  2 01:03:25 1995... \\
We will install a new disk.  System should be back on-line in three hours.  The
system is going DOWN for system halt in 10 minutes !!   } \vspace*{0.5cm} } } 




El mensaje de advertencia (warning)
se repite automáticamente varias veces antes de pagar la máquina con
intervalos cada vez más frecuentes.

 Después del tiempo de postergación, cuando el proceso de apagado real
empieza, se desmontan todos los sistemas de archivos (excepto el sistema de
archivos raíz /), se finalizan (kill) los procesos de los usuarios (si existen
aún usuarios dentro del sistema), los demonios y finalmente se desmonta el
sistema de archivos raíz. Cuando todo este proceso finaliza,
\texttt{\textbf{INIT}} muestra un mensaje indicando que se puede apagar la
máquina. Es entonces y sólo entonces que es posible bajar el switch (o
interruptor de suministro eléctrico).

 Algunas veces, aunque es raro en un buen sistema, es imposible concluir
el sistema de forma adecuada. Por ejemplo, si ocurre un error fatal con el
kernel, puede ser imposible ejecutar cualquier comando nuevo, haciendo la
finalización normal del sistema inviable. Todo lo que se puede hacer en este
caso es esperar que ningún daño severo ocurra y entonces desconectar la máquina.
Si los problemas son menos serios (digamos, alguien quebró su teclado con un
hacha!), y el kernel y el programa \texttt{\textbf{update}} se encuentran en
ejecución normal, es aconsejable aguardar algunos minutos para que
\texttt{\textbf{update}} tenga la chance de actualizar los discos con la
información contenida en el buffer caché, y solamente después desconectar el
equipo.  

 A algunas personas les gusta apagar el equipo tecleando el comando
\texttt{\textbf{sync}}
		\footnote{\texttt{\textbf{sync}} vacía el buffer caché}
		tres veces, dando tiempo a que
		finalice la entrada y la salida del disco duro y entonces apagar
		el equipo. Si no hay programas en ejecución, este método es
		similar a utilizar el comando \texttt{\textbf{shutdown}}. Sin
		embargo, este procedimiento no desmonta los sistemas de archivos
		y puede acarrear problemas con la marca de "limpio" que algunos
		sistemas de archivos utilizan (\textit{filesystem clean
		flag}), como por ejemplo ext2. El método del triple sync
		\textit{no es recomendable}.  

 Por pura curiosidad: la razón del triple \texttt{\textbf{sync}} viene de
los días en que UNIX era joven, cuando los comandos fueron tipeados
separadamente, y eso daba usualmente tiempo suficiente para que la
entrada/salida a disco finalizara.  



\section{ Reinicio (Rebooting)}

 Rebooting significa iniciar el sistema nuevamente. Se puede realizar
finalizando el sistema, apagando la máquina y entonces encendiendo de vuelta. Un
método más práctico para tener el mismo efecto consiste en indicarle al comando
\texttt{\textbf{shutdown}} que reinicie el sistema, en vez de que lo detenga.
La opción \texttt{-r} del comando \texttt{\textbf{shutdown}} puede
realizar la acción anterior, y puede utilizarse inmediatamente, por ejemplo:
\texttt{\textbf{shutdown -r now}}.  

 La mayoría de los sistemas Linux ejecutan el comando \texttt{\textbf{shutdown -r
now}} cuando se oprime la combinación de teclas ctrl-alt-del (y
reinician el equipo). La manera de operar de ctrl-alt-del se puede configurar y
es una buena idea postergar la acción durante algún tiempo antes de reiniciar
una máquina multiusuario. Los sistemas en los que cualquier persona puede
acceder físicamente es conveniente configurar ctrl-alt-del para que no haga
nada.

En algunos Linux también logramos el mismo efecto con el comando reboot.





\section{ Modo usuario individual (single user mode)}

 El comando \texttt{\textbf{shutdown}} también se puede utilizar para
cambiar el nivel de ejecución del sistema al nivel de modo de usuario individual
(single user), en el cual nadie puede iniciar una sesión de usuario, pero root
puede utilizar la consola. Es útil para efectuar tareas de administración del
sistema que no se pueden realizar cuando el sistema se ejecuta normalmente.




\section{ Disquetes de arranque para emergencias}


 No siempre es posible iniciar el sistema desde el disco duro. Por
ejemplo, un error en la configuración de LILO, puede ocasionar que sea imposible
iniciar el sistema. En esas situaciones, es necesario tener una manera
alternativa de iniciar el sistema que funcione siempre (si es que no es el
hardware la causa del problema de inicialización). Para las computadoras de tipo
PC, esta alternativa es iniciar el sistema desde la unidad de disquete.  

 La mayoría de las distribuciones de Linux permiten la creación de
\textit{un disco de arranque de emergencia} (emergency boot
floppy) al momento de efectuar la instalación del sistema. Es conveniente
aprovechar esta opción. Sin embargo, se debe tener en cuenta que algunos de
estos disquetes de arranque solo contienen el kernel y presuponen que empleará
los programas que vienen en los discos de instalación para resolver el problema.


 Algunas veces eso puede no ser suficiente, por ejemplo si necesita
recuperar archivos desde copias de seguridad realizadas con programas que no
vienen en los discos de instalación.  

 En estos casos es necesario crear uno o varios disquetes con un sistema
de archivo raíz personalizado (y que contenga todas las utilidades necesarias
para casos de emergencia). El HOWTO Bootdisk de Graham Chapman contiene
instrucciones de como hacerlos. Es importante también, recordar que los
disquetes de arranque generados para casos de emergencias deben estar siempre
actualizados.  

 No se puede utilizar la unidad de discos flexibles para para montar el
disquete con el sistema de archivos raíz (root filesystem root /) si esa misma
unidad es la que utiliza para iniciar el sistema. Esto es un problema cuando
sólo contamos con una sola unidad de disquete. Sin embargo, si se tiene
suficiente memoria, es posible configurar el diskette de arranque para que se
cargue en un disco virtual en memoria (ramdisk). Cuando el disquete de arranque
ha sido cargado en el disco virtual, la unidad de diskette es liberada y puede
ser utilizada para montar otros disquetes.  

\end{document}
