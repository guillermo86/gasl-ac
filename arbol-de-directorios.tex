
% Custom headers and footers
\usepackage{fancyhdr}
	\pagestyle{fancy}
\usepackage{lastpage}	
	\lhead{}
	\chead{}
	\rhead{}
	\lfoot{\footnotesize Aprendiendo Vim Progresivamente - Departamento de Ingeniería de Computadoras}
	\cfoot{}
	\rfoot{\footnotesize \thepage\ }	% "Page 1 of 2"
	\renewcommand{\headrulewidth}{0.0pt}
	\renewcommand{\footrulewidth}{0.4pt}

\def\ti#1#2{\texttt{#1} & #2 \\ }



\begin{document}

\setlength{\parindent}{0pt}


% Titulo 
\makeatletter
{\color{bl} \centering \huge \sc \textbf{
Aprendiendo Vim Progresivamente \\ 
\large \vspace*{-8pt} \color{black} Vim el Editor de Seis Billones de Dólares
\footnote{http://www.vim.org}
\vspace*{8pt} }\par}
\makeatother


% Autor
\makeatletter
{\centering \small 
	Departamento de Ingeniería de Computadoras \\
	Facultad de Informática - Universidad Nacional del Comahue \\
	\vspace{20pt} }
\makeatother


\section{Visión General del Árbol de Directorios}


\textit{Dos días mas tarde, estaba Pooh sentado en su rama,
balanceando sus patas, y allí junto a él había cuatro ollas de miel}(A.A.
Milne)

 En este capítulo se describen las partes importantes de un árbol de
directorios GNU/Linux estándar, basado en el Estándar de la Jerarquía del
Sistema de Archivos de Linux (Filesystem Hierarchy Standard, FHS). Además, se
explica en líneas generales la forma normal de dividir el árbol de directorios
en  sistemas de archivos separados con diferentes propósitos y se enuncian los
motivos para esta particular división. También se describirán otras formas
alternativas de realizarla.  

\section{ Información preliminar}

 Este capítulo está basado en el \textit{Estándar de la Jerarquía del
Sistema de Archivos de Linux} (FHS) versión 2.1, el cual intenta
establecer un estándar para la organización del árbol de directorios en un
sistema GNU/Linux. Tal estándar tiene la ventaja de facilitar el trabajo de
escribir o portar software a este sistema operativo y administrar máquinas bajo
el mismo, puesto que todas las cosas se encontrarán en lugares estandarizados.
No existe autoridad que obligue a nadie a cumplir con el estándar, pero este
tiene el apoyo de muchas distribuciones GNU/Linux. No es una buena idea romper
con el FHS sin que existan justificaciones indiscutibles. El FHS pretende seguir
la tradición UNIX y las tendencias actuales, haciendo así que los sistemas
GNU/Linux les sean familiares a quienes tengan experiencia con otros sistemas
Unix, y viceversa.  

 Este capítulo no es tan detallado como el FHS. Un administrador de
sistemas debe leer el FHS completo para entenderlo totalmente.  

 En este capítulo no se explican todos los archivos en detalle. La
intención no es describir cada uno de ellos, sino dar una visión general del
sistema desde el punto de vista del sistema de archivos.  Se puede encontrar
información adicional sobre cada archivo en otras partes de este manual o en las
páginas de manual de GNU/Linux.  

 El árbol de directorios completo está pensado para poder ser dividido en
partes más pequeñas, que pueden estar en su propio disco o partición y
acomodarse así a los límites del tamaño del disco, así como para facilitar la
realización de  copias de seguridad y otras tareas de la administración de
sistemas. Las partes principales son los sistemas de archivos raíz
(\texttt{/}) ,\texttt{ /usr},\texttt{ /var
}, y \texttt{/home}. Cada parte tiene un propósito
diferente. El árbol de directorios se ha diseñado para funcionar bien en una red
de máquinas GNU/Linux, las cuales pueden compartir algunas partes del sistema de
archivos sobre un dispositivo de solo-lectura (CD-ROM por ejemplo), o sobre la
red a través de NFS.  


%<figure id="fstree" float="1"> Partes de un árbol de directorios Unix.
%Las líneas discontinuas indican los límites de la partición.  } 
%kk<graphic
%fileref="figuras/fstree.png"/> </figure>


 Los roles de las diferentes secciones del árbol de directorios se
describen a continuación.  

\begin{itemize} \item{ El sistema de archivos raíz es específico para
cada máquina (generalmente se encuentra almacenado en el disco local, aunque
puede estar también en un disco RAM o en una unidad de red) y contiene los
archivos que son necesarios para arrancar el sistema y dejarlo en un estado en
el que se puedan montar los demás sistemas de archivos. El contenido del sistema
de archivos raíz es por lo tanto suficiente para el nivel de ejecución de
usuario individual. También contiene herramientas para reparar un sistema dañado
y para recuperar archivos perdidos desde las copias de seguridad.
}

\item{ El sistema de archivos /usr contiene todos los comandos,
librerías, páginas de manual, y otros archivos que no serán modificados durante
el funcionamiento normal del sistema. No deben existir archivos bajo /usr que
sean específicos para una máquina en particular, ni que deban ser modificados
durante la utilización normal del sistema. Esto permite que los archivos sean
compartidos a través de la red, lo cual puede ser efectivo en cuanto a costes,
puesto que se obtiene un ahorro de espacio en disco (/usr puede ocupar
fácilmente miles de megabytes) y puede facilitar la administración, ya que sólo
el /usr maestro necesita ser modificado cuando actualizamos una aplicación, y no
en cada máquina por separado. Aún cuando el sistema de archivos resida en el
disco local, este puede ser montado en modo solo lectura, para eliminar el
riesgo de que se corrompa durante un fallo.  }

\item{  El sistema de archivos \texttt{/var} contiene
archivos que sí cambian durante el funcionamiento normal del sistema, tales como
directorios spool ( para correo, noticias (news), impresoras, etc), archivos de
log, páginas de manual formateadas y archivos temporales.  Tradicionalmente,
todo en \texttt{/var} es algo que debería estar en
\texttt{/usr}, pero que haría imposible montar dicho
sistema de archivos como solo lectura. }

\item{ El sistema de archivos /home contiene los directorios
específicos de los usuarios, P.Ej., todos los datos reales del sistema. Separar
los directorios home a su propio árbol de directorios o sistema de archivos hace
más fácil la tarea de realizar copias de seguridad; los demás sistemas de
archivos no necesitan que se les haga copias de seguridad, o al menos no tan
frecuentemente, puesto que  rara vez cambian. Un gran directorio /home puede ser
dividido en varios sistemas de archivos, lo cual requiere agregar niveles de
nombres extra, como por ejemplo, /home/estudiantes y /home/staff.
}

\item{ Si bien las diferentes partes del árbol de directorios se han
llamado hasta ahora sistemas de archivos, no se requiere necesariamente que se
encuentren en particiones separadas. Se pueden mantener fácilmente en una única
partición si se trata de un sistema pequeño de un solo usuario, y este sólo
desea mantener las cosas de manera simple. El árbol de directorios puede también
ser dividido en  diferentes particiones dependiendo del tamaño de los discos, y
de como el espacio se destine a los distintos propósitos. Lo importante, no
obstante, es que todos los nombres estándar funcionen; Aún cuando,
digamos,\texttt{/var} y \texttt{/usr} se encuentren de
hecho en la misma partición, los nombres \texttt{/usr/lib/libc.a} y
\texttt{/var/log/messages} deben funcionar. Incluso si, por ejemplo,
moviéramos los archivos que se encuentren en \texttt{/var} dentro de
\texttt{/usr/var}, y hagamos a \texttt{/var} un enlace
simbólico a \texttt{/usr/var}.  } \end{itemize}

 La estructura del sistema de archivos en UNIX agrupa a los archivos de
acuerdo a su propósito. Por lo tanto, todos los comandos están en un mismo
lugar, todos los archivos de datos en otro, la documentación en un tercer lugar,
etc.  Otra alternativa podría ser la de agrupar los archivos de acuerdo al
programa al que pertenezcan, P.Ej., todos los archivos de Emacs podrían
colocarse en un mismo directorio, todos los de Tex en otro, etc. El problema con
esta última aproximación es que dificulta compartir archivos (el directorio del
programa frecuentemente contiene archivos no cambiantes y compartibles, y
cambiantes y no compartibles), y algunas veces incluso encontrar archivos (por
ejemplo, las páginas de manual se encuentran ubicadas en una gran cantidad de
lugares, y hacer que los programas que leen tales páginas de manual las
encuentren sería una pesadilla de mantenimiento).  




\section{ El sistema de archivos raíz}

 El sistema de archivos raíz debería ser pequeño, ya que residen archivos
muy críticos. Si el sistema de archivos es pequeño y rara vez es modificado,
tiene más posibilidades de no sufrir daños. Un sistema de archivos raíz dañado,
generalmente significa que el sistema no podrá arrancar a no ser que se tomen
medidas especiales (por ej., tal vez pueda arrancar desde un disquete de
emergencia), por lo que no se desea correr el riesgo.  

 El directorio raíz no contiene generalmente archivos, exceptuando quizás
la imagen del núcleo estándar, normalmente llamada
\texttt{/vmlinuz}.  Todos los demás archivos se encuentran en
subdirectorios bajo el sistema de archivos raíz:

\begin{itemize} 

	\item \textit{\texttt{/bin}}
	 Comandos necesarios durante el inicio del sistema que
	pueden ser utilizados por usuarios normales (probablemente después de
	que el sistema haya arrancado).  

	\item \textit{\texttt{/sbin}}
	 Igual que \texttt{/bin}, pero aquí los
	comandos 	no están destinados a los usuarios normales,aunque
	pueden utilizarse en caso de que sea necesario y el sistema lo permita.
	\texttt{/sbin} no se encuentra en las rutas de acceso por
	defecto de los usuarios normales. Sí se encuentra definido en la ruta
	por 		defecto para el usuario root.
	


	\item \textit{\texttt{/etc}}
	 Archivos de configuración específicos de la máquina.
	

	
	\item \textit{\texttt{/root}}
	 El directorio local para el usuario root.  normalmente
	los demás usuarios del sistema no pueden acceder a él.
	


	\item \textit{\texttt{/lib}}
	 Librerías compartidas necesarias para los programas que
	se encuentran en el sistema de archivos raíz.
	


	\item \textit{\texttt{/lib/modules}}
	 Módulos cargables del núcleo, especialmente
	aquellos que se necesitan para arrancar el sistema tras recuperarse
	de algún incidente (e.g., controladores de red y sistemas de
	archivos).


	\item \textit{\texttt{/dev}}
	 Archivos de dispositivos. Algunos de los archivos de
	dispositivos más comúnmente utilizados son examinados en el Capítulo 5.
	


	\item \textit{\texttt{/tmp}}
	 Archivos temporales. Los programas que se ejecuten
	después de que el sistema se haya iniciado deben utilizar
	\texttt{/var/tmp}, no \texttt{/tmp},
	debido a que \texttt{/var/tmp} probablemente resida en una
	partición o disco con más espacio. Frecuentemente /tmp es un enlace
	simbólico para /var/tmp.  


	
	\item \textit{\texttt{/boot}}
	 Archivos utilizados por el cargador de arranque, por
	ejemplo, GRUB o LILO. Las imágenes del núcleo se guardan con
	frecuencia en este directorio, en vez de en el directorio raíz. Si
	existen 		muchas imágenes del núcleo,el directorio puede
	llegar a crecer mucho, por 		lo que es mejor mantener este
	directorio en un sistema de archivos 			separado. Otra
	razón puede ser la de asegurarse de que las imágenes del núcleo se
	encuentren dentro de los primeros 1024 cilindros de un disco IDE.
	

	
	\item \textit{\texttt{/mnt}}
	 Punto de montaje temporal para los sistemas de archivos
	montados por el administrador del sistema. Se supone que los programas
	no deben montar en \texttt{/mnt} automáticamente. Es posible
	que \texttt{/mnt} se encuentre dividido en subdirectorios
	(por ej., 			\texttt{/mnt/dos} puede ser
	el punto de montaje para la unidad de disquete con sistema de archivos
	MS-DOS, y \texttt{/mnt/extra} puede llegar a ser lo mismo
	con un sistema de archivos ext2).  

	
	\item \textit{\texttt{/proc}
	\texttt{/usr} \texttt{/var}
	\texttt{/home} }  Puntos de
	montaje para otros sistemas de archivos.
	

 \end{itemize}  




\section{ El directorio /etc}

 El directorio \texttt{/etc} contiene gran cantidad de
archivos. Algunos de ellos se describen aquí, mas abajo. Para otros archivos, se
debe determinar a que programa pertenecen y leer la página de manual
correspondiente.  Muchos archivos de configuración de red se encuentran también
en \texttt{/etc}, y se encuentran descritos en La Guía para
Administradores de Redes en Linux.  

\begin{itemize} 

\item \textit{\texttt{/etc/rc} o
\texttt{/etc/rc.d} o \texttt{/etc/rc?.d}}
 Scripts o directorios de scripts que se ejecutan durante el
arranque del sistema o  al cambiar el nivel de ejecución. Se puede
encontrar información adicional en el capítulo dedicado a Init.


\item \textit{\texttt{/etc/passwd}}
 La base de datos de los usuarios, que incluye campos como el
nombre de usuario, nombre real, directorio home, password encriptada y otra
información acerca de cada usuario.  El formato de este archivo se encuentra
documentado en la página de manual del comando \texttt{\textbf{passwd}}. 
	 Sin embargo, hoy día es muy común encontrar las contraseñas encriptadas
	 en \texttt{/etc/shadow}. Esto significa que en tal caso,
	 los 	 	 datos de los usuarios excepto la password encriptada se
	 encontrarían 	 		 almacenados en
	 \texttt{passwd}.  

	\item \textit{\texttt{/etc/fdprm}}
	 Tabla de parámetros para los discos flexibles.  Describe
	cómo son los diferentes formatos de estos discos.  Este archivo es
	utilizado por el programa \texttt{\textbf{setfdprm}}.  Se puede
	encontrar información adicional en la página de manual de
	\texttt{\textbf{setfdprm}}.  

	\item \textit{\texttt{/etc/fstab}}
	 Lista los sistemas de archivos montados automáticamente
	en 		el arranque del sistema por el comando \texttt{\textbf{mount
	-a}} (en \texttt{/etc/rc} o archivo de inicio
	equivalente). En 			Linux, este archivo también
	contiene información acerca de áreas de swap 		utilizadas
	automáticamente por swapon -a. Se puede encontrar información
	adicional en 
%<xref linkend="mount-and-umount"/>, 
la página de manual del
	comando \texttt{\textbf{mount}}.  

	
	\item \textit{\texttt{/etc/group}}
	 Este archivo es similar a
	\texttt{/etc/passwd} , pero describe grupos en vez de
	usuarios. Se puede encontrar información 		adicional en la
	página de manual del comando \texttt{\textbf{group}}.
	


	\item \textit{\texttt{/etc/inittab}}
	 Archivo de configuración para init.
	


	\item \textit{\texttt{/etc/issue}}
	 Archivos que utiliza \texttt{\textbf{getty}} como
	salida antes de que el sistema pida el nombre de usuario. Usualmente
	contiene una descripción corta o mensaje de bienvenida al sistema.  El
	contenido es establecido por el administrador del sistema.
	


	\item \textit{\texttt{/etc/magic}}
	 El archivo de configuración para el programa
	\texttt{\textbf{file}}. Contiene las descripciones de varios formatos
	de archivos que utiliza file para determinar el tipo de archivo. Se
	puede encontrar información adicional en las páginas de manual
	para \texttt{magic} y \texttt{\textbf{file}}.
	


	\item \textit{\texttt{/etc/motd}}
	 Contiene el mensaje del día, que se emite
	automáticamente  		tras iniciar una sesión con éxito. El
	contenido 	es definido por el 		administrador del
	sistema. Con frecuencia se utiliza para dar información a 	todos
	los usuarios, como por ejemplo, mensajes de advertencias acerca de la
	hora en que está planeada una parada técnica del servidor.
	


	\item \textit{\texttt{/etc/mtab}}
	 Contiene un listado de los sistemas de archivos
	actualmente montados. Se establece Inicialmente por los scripts del
	arranque del sistema, y se actualiza automáticamente por el comando
	\texttt{\textbf{mount}}. Se utiliza cuando se necesita un listado de
	los sistemas de archivos que estén actualmente montados (por ejemplo por
	el comando df).  


	\item \textit{\texttt{/etc/shadow}}
	 Archivo de contraseñas ocultas en sistemas donde se
	encuentre instalado el software de contraseñas ocultas.  Al utilizar
	contraseñas ocultas la password encriptada de cada usuario es eliminada
	de \texttt{/etc/passwd} y colocada en el archivo
	\texttt{/etc/shadow}; este último no 		puede
	ser leído por nadie a excepción del usuario root. De esta manera se
	dificulta el proceso de descifrado de las contraseñas de los usuarios.
	Si 		la distribución GNU/Linux que estemos utilizando nos
	permite elegir 			utilizar o no contraseñas ocultas
	(muchas lo hacen), está altamente 			recomendado
	hacerlo.  


%	\item \textit{\texttt{/etc/login.defs}}
%	 Archivo de configuración para el comando login.  El
%	archivo \texttt{login.defs} se describe en el capítulo 5.
	

	
	\item \textit{\texttt{/etc/printcap}}
	 Similar a \texttt{/etc/termcap}, con la
	excepción de que está destinado a la configuración de colas de
	impresión. La sintaxis también es diferente.  printcap se describe en el
	capitulo 5.   
	

	
	\item \textit{\texttt{/etc/profile},
	\texttt{/etc/csh.login},
	\texttt{/etc/csh.cshrc}}  Archivos
	que se ejecutan en el momento de iniciar los intérpretes de comandos C o
	Bourne. Permite al administrador del sistema establecer parámetros
	globales por defecto para todos los usuarios. Se puede encontrar
	información adicional 	en las páginas de manual para los respectivos
	intérpretes de comandos.  



	\item \textit{\texttt{/etc/securetty}}
	 Identifica las terminales seguras, esto es, las
	terminales por las cuales el usuario root tiene permitido iniciar una
	sesión. Típicamente sólo las consolas virtuales se encuentran listadas
	en este archivo, con lo que se hace imposible (o al menos mas difícil)
	obtener privilegios de superusuario accediendo a través de un módem o la
	red. No se debe permitir iniciar una sesión como usuario root desde la
	red. Es preferible iniciar una sesión con un usuario sin privilegios y
	utilizar después \texttt{\textbf{su}} o \texttt{\textbf{sudo}} para
	obtener privilegios de superusuario.  



	\item \textit{\texttt{/etc/shells}}
	 Listado de intérpretes de comandos admitidos.  El
	comando \texttt{\textbf{chsh}} permite a los usuarios cambiar su
	intérprete de comandos por defecto a otro que se encuentre listado en
	este archivo. \texttt{\textbf{Ftpd}}, el proceso servidor que
	proporciona 	servicios FTP en una máquina, comprueba que los
	intérpretes de comandos 		de los usuarios estén listados
	en \texttt{/etc/shells} y no 		permite que
	nadie inicie una sesión si el intérprete de comandos no se encuentra en
	dicho listado.  

	
	\item \textit{\texttt{/etc/termcap}}
	 La base de datos de capacidades del terminal. Describe
	las \textit{secuencias de escape} por medio de 		las
	cuales se pueden controlar diversos tipos de terminales. Los programas
	se escriben para que, en lugar de generar directamente una secuencia de
	escape que solo funcione en un tipo de terminal, busquen la secuencia
	correcta para hacer lo que necesiten en
	\texttt{/etc/termcap}. 		Como resultado, la
	mayoría de los programas trabajan con la mayoría de los
	tipos de terminales existentes. Se puede encontrar información adicional
	en 	las páginas de manual de termcap, curs\_termcap, y de terminfo.
	

 \end{itemize} 
    



\section{ El directorio /dev}

El directorio \texttt{/dev} contiene los archivos de
dispositivos especiales para todos los dispositivos hardware. Los archivos de
dispositivos se nombran utilizando convenciones especiales; y se describen
con mayor detalle 
en el 
%<xref linkend="device-list"/>
. Los archivos de dispositivos se crean durante
la instalación del sistema, y también pueden ser creados con el script
\texttt{\textbf{/dev/MAKEDEV}}. \texttt{\textbf{/dev/MAKEDEV.local}} es un
script escrito por el administrador del sistema que crea archivos de
dispositivos locales o enlaces (es decir, aquellos que no son parte del
\texttt{\textbf{MAKEDEV}} estándar, como los archivos de dispositivos para algún
controlador de dispositivo no estándar).  




\section{ El sistema de archivos /usr}

El sistema de archivos \texttt{/usr} es con frecuencia grande,
debido a que todos los programas están instalados allí. Normalmente, todos los
archivos en \texttt{/usr} provienen de la distribución Linux que
hayamos instalado; los programas instalados localmente y algunas otras cosas
se encuentran bajo \texttt{/usr/local}. De esta manera es posible
actualizar el sistema desde una nueva versión de la distribución, o incluso
desde una distribución completamente nueva, sin tener que instalar todos los
programas nuevamente. Algunos de los directorios de \texttt{/usr}
están explicados aquí debajo (algunos de los menos importantes se han omitido,
se puede encontrar información adicional en el Estándar del Sistema de
Ficheros).  


\begin{itemize} 
	
	\item \textit{\texttt{/usr/X11R6}}
	 Se encuentran aquí todos los archivos del Sistema
	X-Windows. Para simplificar el desarrollo y la instalación de X,  sus
	archivos no fueron integrados dentro del resto del sistema. Existe un
	árbol de directorios bajo /usr/X11R6 similar al que está bajo /usr.
	


	\item \textit{\texttt{/usr/bin}}
	 En este directorio se encuentran la gran mayoría de los
	comandos para los usuarios. Algunos otros comandos pueden encontrarse en
	\texttt{/bin} o en 	\texttt{/usr/local/bin}.
	

	\item \textit{\texttt{/usr/sbin}}
	 Comandos para la administración del sistema que no son
	necesarios en el sistema de archivos raíz, como por ejemplo la mayoría
	de los programas que proveen servicios.  

	\item \textit{\texttt{/usr/share/man},
	\texttt{/usr/share/info},
	\texttt{/usr/share/doc}}  Páginas
	de manual, documentos de información GNU, y 		archivos de
	documentación de los programas instalados.
	

	\item \textit{\texttt{/usr/include}}
	 Archivos cabecera para el lenguaje de programación C.
	Estos deberían estar de hecho debajo de \texttt{/usr/lib}
	por coherencia, pero tradicionalmente se ha apoyado de forma mayoritaria
	esta ubicación.  

	\item \textit{\texttt{/usr/lib}}
	 Archivos de datos de programas y subsistemas que no
	sufren cambios, incluyendo algunos archivos de configuración globales.
	El 		nombre lib viene de librería; originariamente las
	librerías de las 			subrutinas de programación se
	almacenaban en \texttt{/usr/lib}.
	

	\item \textit{\texttt{/usr/local}}
	 Es la ubicación para el software instalado localmente y
	para algunos otros archivos.  Las distribuciones no deben colocar
	archivos bajo este directorio. Se 			reserva para ser
	utilizado únicamente por el administrador local del
	sistema. De esta manera, aquel se asegura totalmente de que ninguna
	actualización de su distribución sobreescribirá  el software que él
	mismo haya instalado localmente.  


 \end{itemize} 




\section{ El sistema de archivos /var}
    
    El sistema de archivos \texttt{/var} contiene datos que
    cambian cuando el sistema se ejecuta normalmente. Es específico para cada
    sistema y por lo tanto no es compartido a través de la red con otras
    computadoras. 

	\begin{itemize}  

	\item 
        
	\textit{\texttt{/var/cache/man}} 
        
	 Actúa como una caché para las páginas de manual que son
	formateadas bajo demanda. Las fuentes de las páginas de manual se
	encuentran almacenadas normalmente en \texttt{/usr/man/man?}
	(donde ? es la sección de las páginas de manual que corresponda. Se
	puede 	encontrar información adicional en la página de manual para
	\texttt{\textbf{man}} en el capítulo 7); algunas páginas de manual
	pueden 	llegar a venir con una versión pre-formateada, la cual estaría
	almacenada 	en \texttt{/usr/share/man/cat*}. Otras
	páginas de manual necesitan ser formateadas al ser visualizadas por
	primera vez; la versión formateada es almacenada entonces en
	\texttt{var/cache/man} para que la próxima vez que un
	usuario necesite ver la misma página no tenga que esperar a que se le de
	formato.  

	\item 
        
	\textit{\texttt{/var/games}} 
        
	 Cualquier información variable perteneciente a juegos
	existente en \texttt{/usr} debería colocarse aquí. Esto es
	así por si se da el caso de que \texttt{/usr} esté montado
	como 		solo lectura 

	\item 

	\textit{\texttt{/var/lib}} 
    
	 Contiene archivos que cambian mientras el sistema se
	ejecuta normalmente. 

	\item 
        

	\textit{\texttt{/var/local}} 
        
	 Datos variables de los programas que se encuentran
	instalados en \texttt{/usr/local}(aquellos que fueron
	instalados localmente por el administrador del sistema). Conviene
	reseñar  		que aunque los programas se encuentren
	instalados localmente, deben 			utilizar los otros
	directorios \texttt{/var} en caso de ser
	apropiado, como por ejemplo:
	\texttt{/var/lock}.  

	\item 
        
	\textit{\texttt{/var/lock}} 
        
	 Archivos de bloqueo. Muchos programas siguen una
	convención para crear un archivo de bloqueo en /var/lock que indique que
	están utilizando un dispositivo particular o un archivo de forma
	exclusiva. Así, Los demás programas se encontrarán con el archivo de
	bloqueo y no intentarán acceder al dispositivo o archivo.
	  

	\item 

	\textit{\texttt{/var/log}} 

	  Archivos de log (en español bitácora) de diferentes
	 programas, especialmente de \texttt{\textbf{login}}
	 (\texttt{/var/log/wtmp}, el cual registra todos los inicios
	 y cierres de sesión en el sistema) y de \texttt{\textbf{syslog}}
	 (\texttt{/var/log/messages}, en donde se almacenan todos
	 los mensajes del núcleo y de los programas del sistema). Los Archivos
	 dentro del directorio \texttt{/var/log} pueden crecer
	 indefinidamente, por lo que se requiere una limpieza a intervalos
	 regulares.   

	\item 
        
	\textit{\texttt{/var/mail}} 
        
	 Esta es la ubicación definida por el FHS (Estándar de la
	jerarquía del sistema de ficheros) para almacenar los archivos de buzón
	de correos de los usuarios. Dependiendo de en qué 		grado su
	distribución cumpla con el FHS, estos archivos pueden 	llegar a
	estar ubicados en \texttt{/var/spool/mail}.
	  

	\item 

	\textit{\texttt{/var/run}} 
        
	 Directorio que contiene archivos con información acerca
	del sistema, la cual es válida hasta el próximo inicio del mismo. Por
	ejemplo: \texttt{/var/run/utmp} contiene información de las
	personas que actualmente tienen sesiones iniciadas.  
	 

	\item 
        
	\textit{\texttt{/var/spool}} 
        
	 Contiene directorios para las noticias, el correo, colas
	de impresión, y otros programas que necesiten trabajar con colas.
	Cada spool diferente tiene su propio directorio debajo de
	\texttt{/var/spool}, por ejemplo: el spool de noticias se
	encuentra en \texttt{/var/spool/news}. Cabe destacar que si
	alguna instalación no cumple totalmente con la última versión del
	FHS, los buzones de correo entrante de los usuarios pueden encontrarse
	en /var/spool/mail.   

	\item 
        
	\textit{\texttt{/var/tmp}} 
        
	 Archivos temporales grandes o que necesitan existir por
	un tiempo más amplio de lo permitido en \texttt{/tmp}.  (De
	todas formas, el administrador del sistema puede no permitir muchos
	archivos antiguos en /var/tmp si así lo desea).  
	  \end{itemize}  


\section{ El sistema de archivos /proc}

El sistema de archivos \texttt{/proc} contiene un sistema de
archivos imaginario o virtual. Este no existe físicamente en disco, sino que el
núcleo lo crea en memoria. Se utiliza para ofrecer información relacionada con
el sistema (originalmente acerca de procesos, de aquí su nombre). Algunos de los
archivos más importantes se encuentran explicados mas abajo. El sistema de
archivos \texttt{/proc} se encuentra descrito con más detalle en la
página de manual de proc.

	\begin{itemize} 
	
	\item

	\textit{\texttt{/proc/1}}

	 Un directorio con información acerca del proceso número
	1. 	Cada proceso tiene un directorio debajo de /proc cuyo nombre es
	el número de identificación del proceso (PID).  
	
	
	\item
	
	\textit{\texttt{/proc/cpuinfo}}
	
	 Información acerca del procesador: su tipo, marca,
	modelo, 	rendimiento, etc.  
	
	
	\item
	
	\textit{\texttt{/proc/devices}}
	
	 Lista de controladores de dispositivos configurados
	dentro 	del núcleo que está en ejecución.
	
	
	\item \textit{\texttt{/proc/dma}}
	 Muestra los canales DMA que están siendo utilizados.
	
	
	\item
	\textit{\texttt{/proc/filesystems}}
	 Lista los sistemas de archivos que están soportados por
	el kernel.  
	
	\item
	\textit{\texttt{/proc/interrupts}}
	 Muestra la interrupciones que están siendo utilizadas, y
	cuantas de cada tipo ha habido.  
	
	\item \textit{\texttt{/proc/ioports}}
	 Información de los puertos de E/S que se estén
	utilizando 	en cada momento.  
	
	\item \textit{\texttt{/proc/kcore}}
	 Es una imagen de la memoria física del sistema. Este
	archivo tiene exactamente el mismo tamaño que la memoria física, pero no
	existe en memoria como el resto de los archivos bajo /proc, sino que se
	genera en el momento en que un programa accede a este. (Recuerde: a
	menos que copie este archivo en otro lugar, nada bajo
	\texttt{/proc} usa espacio en disco).
	
	
	\item
	
	\textit{\texttt{/proc/kmsg}}
	
	 Salida de los mensajes emitidos por el kernel. Estos
	también son redirigidos hacia \texttt{\textbf{syslog}}.
		
	

	\item 

	\textit{\texttt{/proc/ksyms}}
	 Tabla de símbolos para el
	kernel.
	
	\item \textit{\texttt{/proc/loadavg}}
	 El nivel medio de carga del sistema; tres indicadores
	significativos sobre la carga de trabajo del sistema en cada
	momento.
	
	\item \textit{\texttt{/proc/meminfo}}
	 Información acerca de la utilización de la memoria
	física y 	del archivo de
	intercambio.
	
	\item
	
	\textit{\texttt{/proc/modules}}
	
	 Indica los módulos del núcleo que han sido cargados
	hasta el 	momento.
	

	\item \textit{\texttt{/proc/net}}         
    
	 Información acerca del estado de los protocolos de
	red.
	
	\item
	
	\textit{\texttt{/proc/self}}
	
	 Un enlace simbólico al directorio de proceso del
	programa que esté observando a \texttt{/proc}. Cuando dos
	procesos observan a \texttt{/proc}, obtienen diferentes
	enlaces. Esto es principalmente una conveniencia para que sea fácil para
	los programas acceder a su directorio de
	procesos.
	
	\item \textit{\texttt{/proc/stat}}
	 Varias estadísticas acerca del sistema, tales como el
	número de fallos de página que han tenido lugar desde el arranque del
	sistema.
	
	\item \textit{\texttt{/proc/uptime}}
	 Indica el tiempo en segundos que el sistema lleva
	funcionando.
	
	\item \textit{\texttt{/proc/version}}
	 Indica la versión del núcleo
	
	
	 \end{itemize} 

    Conviene aclarar que aunque los archivos anteriores tienden a ser
    archivos de texto fáciles de leer, algunas veces pueden tener un formato
    que no sea fácil de interpretar. Por ello existen muchos comandos que
    solamente leen los archivos anteriores y les dan un formato distinto para
    que la información sea fácil de entender. Por ejemplo, el comando
    \texttt{\textbf{free}}, lee el archivo \texttt{/proc/meminfo}
    y convierte las cantidades dadas en bytes a kilobytes (además de agregar un
    poco más de información extra). 





\end{document}



