%%% LaTeX Template: Article/Thesis/etc. with colored headings and special fonts
%%%
%%% Source: http://www.howtotex.com/

\documentclass[12pt]{article}


\usepackage{apuntes-estilo}
\usepackage{fancyhdr,lastpage}



\def\maketitle{

% Titulo 
 \makeatletter
 {\color{bl} \centering \huge \sc \textbf{
Trabajo práctico N 1 \\
\large \vspace*{-8pt} \color{black} Editor de Textos Vim
 \vspace*{8pt} }\par}
 \makeatother


% Autor
 \makeatletter
 {\centering \small 
	Introducción a la administración de Sistemas \\
 	Departamento de Ingeniería de Computadoras \\
 	Facultad de Informática - Universidad Nacional del Comahue \\
 	\vspace{20pt} }
 \makeatother

}

% Custom headers and footers
\fancyhf{} % clear all header and footer fields
\fancypagestyle{plain}{\fancyhf{}}
  	\pagestyle{fancy}
 	\lhead{\footnotesize TP N 1 - Introducción a la administración de sistemas }
 	\rhead{\footnotesize \thepage\ }	% "Page 1 of 2"

\def\ti#1#2{\texttt{#1} & #2 \\ }



\begin{document}

\thispagestyle{empty}
\maketitle
\setlength{\parindent}{0pt}

\paragraph{Los siguientes ejercicios se realizan en su totalidad desde una terminal
en un sistema GNU/Linux}. 

\section*{Repaso}
\begin{enumerate}
\item ¿Qué comando utiliza para saber en cuál directorio se encuentra usted trabajando?
\item ¿Cómo saber cuáles son las opciones soportadas por un comando sin recurrir a Internet?
\item Qué comandos utiliza para: 
	\begin{enumerate}
	\item Listar todos los procesos en ejecución
	\item Listar el contenido del directorio actual con los detalles de cada archivo. 
	\item Cambiar el directorio de trabajo a otro
	\item Observar la jerarquía de procesos en ejecución (mencione al menos dos formas de hacerlo)
	\item Eliminar un archivo
	\item Eliminar un directorio 
	\item Observar el contenido de un archivo de texto
	\item Observar el contenido de un archivo de texto largo, paginando la salida en 
	la terminal (mencione al menos dos maneras de hacerlo). 
	\end{enumerate}
\end{enumerate}

\section*{Vim}
\begin{enumerate}
\item Ejecute la siguiente secuencia:
	\begin{itemize}
	\item \texttt{vim ej1.txt}
        \item Pasar a modo inserción presionando "i"
        \item Escribir tres lineas a elección. 
        \item Volver al modo normal presionando ''ESC''
        \item \texttt{:wq}
	\item Observar el contenido del archivo ej1.txt con el comando \texttt{cat}. 
	\end{itemize}
\item Ejecute la siguiente secuencia:
	\begin{itemize}
        \item \texttt{vim ej1.txt}
        \item Moverse con los cursores a la linea del medio
        \item Ejecutar \texttt{dd}
        \item \texttt{:wq}
	\item Observar el contenido del archivo ej1.txt con el comando \texttt{cat}. 
        \item Explicar qué sucedió. 
	\end{itemize}
\item Instale el plugin de Vim HJKL (copie el archivo hjkl.vim en \textasciitilde/.vim/plugin/). 
 \footnote{http://www.vim.org/scripts/script.php?script\_id=3409}
Abra una terminal, ejecute \texttt{vim} y posteriormente el comando \texttt{:HJKL} en \textit{modo normal}. 
Juegue hasta sentirse cómodo con las teclas hjkl. 
\item Ejecute la siguiente secuencia:
	\begin{itemize}
	\item \texttt{vimtutor}. 
	\item Guarde el contenido de vimtutor en una archivo en su home, ejecutando en modo normal: \texttt{:saveas tutor.\textless hostname\textgreater} (debe reemplazar \textless hostname\textgreater por el nombre de la máquina que está utilizando).
	\item Agregue su nombre y apellido al comienzo del archivo.
	\item Complete las lecciones ofrecidas por el tutor. 
	\item Guarde sus cambios a medida que avanza: \texttt{:w} (verifique el nombre de archivo donde esta guardando sus cambios) 
	\end{itemize}
\end{enumerate}


\end{document}
